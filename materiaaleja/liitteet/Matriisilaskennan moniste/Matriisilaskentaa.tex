\documentclass[a4paper, 12pt]{article}

\usepackage{chemfig}
\makeatletter
%\renewcommand{\@chapapp}{}% Not necessary...
%\newenvironment{chapquote}[2][2em]
 % {\setlength{\@tempdima}{#1}%
 %  \def\chapquote@author{#2}%
 %  \parshape 1 \@tempdima \dimexpr\textwidth-2\@tempdima\relax%
 %  \itshape}
 % {\par\normalfont\hfill--\ \chapquote@author\hspace*{\@tempdima}\par\bigskip}
\makeatother
%\renewcommand\thechapter{\Roman{chapter}}
%\usepackage[paper=a4paper,textwidth=16.5cm,bottom=40mm]{geometry}
\usepackage[left=2.5cm, top=3.0cm,right=2.5cm, bottom=2.5cm]{geometry}
%\usepackage{avant}
%\usepackage[scaled=.92]{helvet}
%\renewcommand*\familydefault{\sfdefault}
\usepackage[utf8]{inputenc}
\usepackage{romannum}
%\usepackage{cmbright}
%\usepackage{ccfonts,eulervm}
%\usepackage{eulervm}
\usepackage[scaled]{helvet}
\renewcommand\familydefault{\sfdefault}
\usepackage[T1]{fontenc}

\usepackage{tocloft}
\usepackage{lipsum}
\usepackage{titlesec}

\newcommand{\secmark}{}
\newenvironment{advanced}
  {\renewcommand{\secmark}{*}}
  {}

\titleformat{\subsection}
  {\large\bfseries} % Format of the title
  {\thesection\secmark} % Label
  {1em} % Separation between label and title body 
        % (default = horizontal space, display = vertical space)
  {} % Code preceding the title

\usepackage[english, finnish]{babel}
\usepackage{tabularx}
\usepackage{ae}
%\usepackage{cleveref}
\usepackage[affil-it]{authblk}
%\usepackage{CV}
\usepackage{url}
\usepackage{xcolor}
\usepackage{amsmath}
\usepackage{amsthm}



\usepackage{ulem}
\usepackage[pdfpagemode=None,linktocpage=true, colorlinks=true,urlcolor=red, linkcolor=black,citecolor=black,pdfstartview=FitH,linktoc=all]{hyperref}
\usepackage{cleveref}
\makeatletter  
\def\@endtheorem{\qed\endtrivlist\@endpefalse } % insert `\qed` macro
\makeatother
%\renewcommand{\qedsymbol}{}
\usepackage{thmtools}
\usepackage{amsthm}
\usepackage{thmtools}
\declaretheoremstyle[headfont=\color{blue}\normalfont\bfseries, bodyfont=\normalfont, headpunct={.}, qed={}]{normalbody}
\declaretheoremstyle[headfont=\color{blue}\normalfont\bfseries, bodyfont=\normalfont, headpunct={.}]{normalbody1}
\declaretheoremstyle[headfont=\normalfont\bfseries, bodyfont=\normalfont, headpunct={.}, qed={}]{normalbody2}
\definecolor{auburn}{rgb}{0.43, 0.21, 0.1}
\declaretheoremstyle[headfont=\color{auburn}\normalfont\bfseries, bodyfont=\normalfont, headpunct={.}, qed={}]{normalbody3}
%\theoremstyle{exercise}
\declaretheorem[style=normalbody,name=Tehtävä]{teht}
%\newtheorem{teht}{Tehtävä}
\declaretheorem[style=normalbody,name=Esimerkki]{esim}
\declaretheorem[style=normalbody2,name={Lause}]{lause}
\declaretheorem[style=normalbody3,name=Määritelmä]{maar}

\declaretheoremstyle[%
  spaceabove=0pt,%
  spacebelow=6pt,%
  headfont=\normalfont\itshape,%
  %postheadspace=1em,%
  %
  headpunct={.},
]{mystyle} 
\declaretheorem[style=normalbody2,name=Ratkaisu, unnumbered]{rat}
\declaretheorem[style=mystyle,name=Todistus, unnumbered]{tod}
%\def\eolqed{\hfill\qedsymbol}
%\newcounter{määr}{section}
\newenvironment{määr}[2]
{\begin{tcolorbox}[adjusted title=Määritelmä #1,colframe=blue!75!black]
%\begin{maar*}[''Jakaa täysin'']
#2
%\end{maar*}
\end{tcolorbox}
}

\declaretheoremstyle[headfont=\normalfont\bfseries, headpunct={.}, qed={}]{normalbody5}
\declaretheorem[style=normalbody5,name=Huomautus, unnumbered]{huom}
\declaretheorem[style=normalbody5,name=Seuraus, unnumbered]{seur}

%\renewcommand{\qedsymbol}{}
%\declaretheorem[style=mythmstyle, name=Lause]{lause}
\theoremstyle{remark}
%\newtheorem*{huom}{Huomautus}
\theoremstyle{definition}
%\newtheorem{theorem}{Lause}[chapter]

\newtheorem*{maar*}{Määritelmä}
%\newtheorem*{seur}{Seuraus}
%\newtheorem*{huom}{Huomautus}

%\newtheorem{esim}{Esimerkki}
%\newtheorem*{rat}{Ratkaisu}
%\newtheorem{lause}{Lause}
%\theoremstyle{exercise}
%\declaretheorem[style=normalbody,name=Tehtävä]{teht}


\usepackage{amssymb}
\usepackage{csquotes}
\usepackage{amsfonts}
\usepackage{mathdesign}
\usepackage{mathtools}
\usepackage{graphicx}
 %\usepackage{caption}
\usepackage{subcaption}
\usepackage{xr-hyper}
%\usepackage[backend=bibtex]{biblatex}   % bibliography

%\usepackage[pdfpagemode=FullScreen, colorlinks=true,urlcolor=red, linkcolor=blue,citecolor=black,pdfstartview=FitH]{hyperref}
\usepackage{url}
%\usepackage{cite}

\usepackage{latexsym}

\usepackage[decimalsymbol=comma, load=prefixed]{siunitx}
%\usepackage{siunitx}
%\sisetup{output-decimal-marker = {,}}
   \sisetup{exponent-product = \cdot, output-product = \cdot}
\sisetup{
    group-digits=true,
    group-separator={\,},
%   group-four-digits=false,% default setting
}
\usepackage{fancyhdr}
%\usepackage{fancyref}
%
% This is not working with package fancyref.

\usepackage{fancyhdr} % heade

% Quotes
\usepackage{epigraph}

% Headerit
\pagestyle{fancyplain}
\lhead[\fancyplain{}{\bfseries\thepage}]
	{\fancyplain{}{\bfseries }}
\chead[\fancyplain{}{\bfseries\thepage}]
	{\fancyplain{}{\bfseries }}
\rhead[\fancyplain{}{\bfseries\leftmark}]
	{\fancyplain{}{\bfseries \dmyyyydate\today}}
%\cfoot{}
%\lfoot{}
%\rfoot{tihayryn@lce.hut.fi}
%
% ei kappaleen sisennystä ja tyhjä rivi kappaleiden väliin
\setlength{\parindent}{0pt}
\setlength{\parskip}{1ex plus 0.5ex minus 0.5ex}
%\setlength{\parskip}{3mm plus0.5mm minus0.5mm}
\usepackage{sectsty}
%\sectionfont{\fontsize{16}{15}\selectfont}
%\paragraphfont{\fontsize{14}{15}\selectfont}
%\usepackage{fouriernc}
%\usepackage{ccfonts,eulervm}

%\usepackage[scaled=.92]{helvet}
\usepackage{datetime}
\renewcommand{\dateseparator}{.}
\date{\dmyyyydate\today}

%\setlength{\textwidth}{160mm}
%\setlength{\oddsidemargin}{-5mm}
%\setlength{\evensidemargin}{-5mm}
%\setlength{\textheight}{255mm}
%\setlength{\topmargin}{-15mm}

\usepackage{natbib}
\bibliographystyle{./Bibliography/my-dcu}
%\bibliographystyle{agsm}
%\bibliographystyle{dcu}
%\bibliographystyle{abbrvnat}
%\bibliographystyle{plainnat}
%\bibliographystyle{unsrtnat}
%\renewcommand\harvardyearleft{\unskip, }
%\renewcommand\harvardyearright[1]{.}

\renewcommand\harvardyearleft{\unskip\ }
\renewcommand\harvardyearright[1]{.}
\bibpunct{(}{)}{;}{a}{}{,~}
\AtBeginDocument{\renewcommand{\harvardand}{ja}}
%\renewcommand\harvardand{ja}
\renewcommand{\arraystretch}{1.3}


\usepackage{amssymb}
%\usepackage{mathdesign}
\usepackage{gauss}
\usepackage{mathabx}

\usepackage{graphicx} % Kuvien lis??????ist??????arten
\usepackage{appendix} % Lis??????vaihtoehtoja liitteiden muotoiluun.


\usepackage{tikz}
\usetikzlibrary{shadings}
\usetikzlibrary{automata, positioning}
\usepackage{tikz-3dplot}
\usetikzlibrary{decorations.fractals}
\usepackage{tikz-3dplot}
\usetikzlibrary{calc, intersections}	       %allows coordinate calculations.
\usetikzlibrary{angles,patterns,calc}
\usepackage{sidecap}
\usetikzlibrary{calc,fadings,decorations.pathreplacing}
\usepackage{xcolor}
\usetikzlibrary{positioning}
\usetikzlibrary{through,backgrounds}
\usepackage{footmisc}
\usepackage{sidecap}
\usetikzlibrary{calc,fadings,decorations.pathreplacing}
\usepackage{xcolor}
\usetikzlibrary{positioning}
\usetikzlibrary{through,backgrounds}
\usepackage{footmisc}
\usetikzlibrary{decorations.pathmorphing,decorations.markings}


\usetikzlibrary{decorations}

% Not sure this follows any proper defition of 
% Sirpinksi triangle. It just works.
\pgfdeclaredecoration{quasi-sirpinski}{do}{%
    \state{do}[width=\pgfdecoratedinputsegmentlength, next state=do]{%
        \pgfpathmoveto{\pgfpointpolar{-60}{\pgfdecoratedinputsegmentlength/2}}%
        \pgfpathlineto{\pgfpointorigin}%
        \pgfpathlineto{\pgfpoint{\pgfdecoratedinputsegmentlength/2}{0pt}}%
        \pgfpathclose%
    }
}
% TIKZ - for drawing Feynman diagrams
% ... use with pdflatex




\usepackage{tikz}
\usetikzlibrary{arrows,shapes}
\usetikzlibrary{trees}
\usetikzlibrary{matrix,arrows} 				% For commutative diagram
			\usepackage{pgfplots}								% http://www.felixl.de/commu.pdf
\usetikzlibrary{positioning}				% For "above of=" commands
\usetikzlibrary{calc,through}				% For coordinates
\usetikzlibrary{decorations.pathreplacing}  % For curly braces
% http://www.math.ucla.edu/~getreuer/tikz.html
\usepackage{pgffor}							% For repeating patterns

\usetikzlibrary{decorations.pathmorphing}	% For Feynman Diagrams
\usetikzlibrary{decorations.markings}
\tikzset{
	% >=stealth', %%  Uncomment for more conventional arrows
    vector/.style={decorate, decoration={snake}, draw},
	provector/.style={decorate, decoration={snake,amplitude=2.5pt}, draw},
	antivector/.style={decorate, decoration={snake,amplitude=-2.5pt}, draw},
    fermion/.style={draw=black, postaction={decorate},
        decoration={markings,mark=at position .55 with {\arrow[draw=black]{>}}}},
    fermionbar/.style={draw=black, postaction={decorate},
        decoration={markings,mark=at position .55 with {\arrow[draw=black]{<}}}},
    fermionnoarrow/.style={draw=black},
    gluon/.style={decorate, draw=black,
        decoration={coil,amplitude=4pt, segment length=5pt}},
    scalar/.style={dashed,draw=black, postaction={decorate},
        decoration={markings,mark=at position .55 with {\arrow[draw=black]{>}}}},
    scalarbar/.style={dashed,draw=black, postaction={decorate},
        decoration={markings,mark=at position .55 with {\arrow[draw=black]{<}}}},
    scalarnoarrow/.style={dashed,draw=black},
    electron/.style={draw=black, postaction={decorate},
        decoration={markings,mark=at position .55 with {\arrow[draw=black]{>}}}},
	bigvector/.style={decorate, decoration={snake,amplitude=4pt}, draw},
}

% TIKZ - for block diagrams, 
% from http://www.texample.net/tikz/examples/control-system-principles/
% \usetikzlibrary{shapes,arrows}
\tikzstyle{block} = [draw, rectangle, 
    minimum height=3em, minimum width=6em]






\usepackage{listing}
\usepackage{wasysym}
\usepackage{bbm}
\usepackage{fancybox}

\usepackage{textcomp}

\usepackage[labelfont=bf,labelsep=period]{caption}

%\usepackage{caption}
\usepackage{subcaption}
\usepackage{wrapfig}


\usepackage[nottoc,notlot,notlof]{tocbibind}
%\usepackage[nottoc]{tocbibind}

\addto\captionsenglish{\renewcommand{\bibname}{References}}
% Mathematics
\newcommand{\vii}{\mathop{\Big/}}
\newcommand{\viiva}[2]{\vii\limits_{\!\!\!\!{#1}}^{\>\,{#2}}}
\newcommand{\relR}{\mathrel{\mathsf{R}}}
\newcommand{\cp}[1]{{#1}^\complement}
\newcommand{\raj}[2][f]{#1\!\!\mid_{#2}}
\renewcommand{\bar}[1]{\overline{#1}}
\newcommand{\vek}[1]{\mbox{\boldmath$#1$}}

\renewcommand{\vec}[1]{\vek{#1}}
\newcommand{\vecp}[1]{\hat{\vec{#1}}}
\newcommand{\matr}[1]{\mathsf{#1}}
\newcommand{\E}{\mathrm{e}}
\newcommand{\I}{\mathrm{i}}
\newcommand{\D}[1][x]{\,\text{d}#1}
%\newcommand{\km}{\,\mathrm{km}}
\newcommand{\abs}[1]{\lvert#1\rvert}
\newcommand{\norm}[1]{\lvert\lvert#1\rvert\rvert}
\newcommand{\mean}[1]{\langle#1\rangle}
\newcommand{\fneg}{f^{-1}}
\newcommand{\ftil}{\tilde{f}}
\newcommand{\ainaa}{{\alpha \in A}}
\newcommand{\INT}[1]{\overset{\mspace{6mu}\circ}{#1}}
\newcommand{\osit}[2]{\frac{\partial #1}{\partial #2}}
\newcommand{\diff}[2]{\frac{d#1}{d#2}}
\newcommand{\difft}[2]{\frac{d^2 #1}{d#2 ^2}}
\newcommand{\ositt}[2]{\frac{\partial^2 #1}{\partial #2^2}}
\newcommand{\epsi}{\varepsilon}
\newcommand{\epsii}{\epsi_{i}}
\newcommand{\epsie}{\epsi_{eff}}
\newcommand{\epsit}{\epsi_{t}}
\newcommand{\epsir}{\epsi_{r}}
\newcommand{\uz}{\vek{u}_z}
\newcommand{\roo}{\varrho}
\newcommand{\fii}{\phi_{i}}
\newcommand{\fit}{\phi_{t}}
\newcommand{\Ei}{\vek{E}_{i}}
\newcommand{\Et}{\vek{E}_{t}}
\newcommand{\pr}{\mathfrak{p}}
\renewcommand{\Re}[1]{\textrm{Re}\, #1}
\renewcommand{\Im}[1]{\textrm{Im}\, #1}
%\renewcommand{\Re}{{\operatorname{Re}}}           % real part
%\renewcommand{\Im}{{\operatorname{Im}}}           % imaginary part
\DeclareMathOperator{\Arg}{Arg}
\DeclareMathOperator{\Ln}{Ln}
\DeclareMathOperator{\diag}{diag}

\newcommand{\phg}{p_\text{HG}}
\newcommand{\kB}{k_\textrm{B}}

\newcommand{\nub}{\bar{\nu}}

\newcommand{\bra}[1]{\ensuremath{\left\langle#1\right|}}
\newcommand{\ket}[1]{\ensuremath{\left|#1\right\rangle}}
\newcommand{\bracket}[2]{\ensuremath{\left\langle#1 \vphantom{#2}\right| \left. #2 \vphantom{#1}\right\rangle}}
\newcommand{\matrixel}[3]{\ensuremath{\left\langle #1 \vphantom{#2#3} \right| #2 \left| #3 \vphantom{#1#2} \right\rangle}}
\newcommand{\ipr}[1]{\ensuremath{\left\langle #1\right\rangle}}


\newcommand\ddfrac[2]{\frac{\displaystyle #1}{\displaystyle #2}}

\newcommand{\ave}[1]{\ensuremath{\left\langle #1 \right\rangle}}
\newcommand{\dotpr}[2]{\ensuremath{#1 \cdot #2}}
%\newcommand{\dyad}[1]{\overleftrightarrow{#1}}
\newcommand{\dyad}[1]{\bar{\bar #1}}

\newcommand{\St}[1]{#1_{S}}
\newcommand{\aSt}[1]{#1_{aS}}
\newcommand{\vib}[1]{#1_{vib}}
\newcommand{\op}[1]{\widehat{#1}}
\newcommand{\Hami}{\mathcal{H}}

\newcommand{\comm}[2]{\ensuremath{\left[ #1,\, #2 \right]}}
\newcommand{\acomm}[2]{\ensuremath{\left\{ #1,\, #2 \right\}}}
%\renewcommand{\v}{\mathrm{v}}

 \usepackage{relsize}
%\newcommand{\comm}[2]{%
 % [%
%  \begin{array}{
 %   @{}
  % >{\centering $\displaystyle}p{1em}<{$}
  % @{,}
  %  >{\centering $\displaystyle}p{1.5em}<{$}
  % @{}
 %}
  % #1 & #2
 %\end{array}%
 %]%
%}

%\newcommand{\acomm}[2]{%
 % \{%
  %\begin{array}{
   % @{}
   %>{\centering $\displaystyle}p{1em}<{$}
   %@{,}
   % >{\centering $\displaystyle}p{1.5em}<{$}
   %@{}
 %}
  % #1 & #2
 %\end{array}%
 %\}%
%}

%\newcommand{\comm}[3][1.2em]{[ \makebox[#1]{$#2$} , \makebox[#1]{$#3$} ]}
%\newcommand{\acomm}[3][1.2em]{\left\{ \makebox[#1]{$#2$} , \makebox[#1]{$#3$} \right\} }
 


\DeclareMathOperator{\sinc}{sinc}
\DeclareMathOperator{\tr}{tr}
\DeclareMathOperator{\Null}{null}

%\usepackage[nottoc]{tocbibind}
\usepackage{enumitem}
\makeatletter
% This command ignores the optional argument for itemize and enumerate lists
\newcommand{\inlineitem}[1][]{%
\ifnum\enit@type=\tw@
    {\descriptionlabel{#1}}
  \hspace{\labelsep}%
\else
  \ifnum\enit@type=\z@
       \refstepcounter{\@listctr}\fi
    \quad\@itemlabel\hspace{\labelsep}%
\fi}
\makeatother

\newenvironment{itemize*}%
 {\begin{itemize}%
    \setlength{\itemsep}{0pt}%
  \setlength{\parskip}{0pt}}%
  {\end{itemize}}
\renewenvironment{itemize*}{\itemize}


\newenvironment{enumerate*}%
  {\begin{itemize}%
    \setlength{\itemsep}{0pt}%
    \setlength{\parskip}{0pt}}%
  {\end{itemize}}
\renewenvironment{enumerate*}{\enumerate}


\usepackage{exsheets}

\newenvironment{listaa} %%%% this adjusts spacing as I desire
    {\begin{enumerate}[leftmargin=*, label=\alph*), topsep=0pt, itemsep=12pt, parsep=9pt, font=\bfseries\color{red}] \itemsep0pt \parskip0pt \parsep0pt \topsep0pt}
    {\end{enumerate}}
    \newenvironment{listab} %%%% this adjusts spacing as I desire
    {\begin{enumerate}[leftmargin=*, label=\roman*), topsep=0pt, itemsep=12pt, parsep=9pt, font=\bfseries\color{blue}] \itemsep0pt \parskip0pt \parsep0pt \topsep0pt}
    {\end{enumerate}}

\usepackage{tcolorbox}% http://ctan.org/pkg/tcolorbox
    \tcbuselibrary{skins,breakable}
\newtcolorbox{mybox}[1]{colback=red!5!white,colframe=red!75!black,fonttitle=\bfseries,title=#1}
\newtcolorbox{myboxi}[1]{colframe=blue!75!black,fonttitle=\bfseries,title=#1}
\newtcolorbox{myboxa}[1]{colframe=yellow!15!black,fonttitle=\bfseries,title=#1}

%\usepackage{cleveref}

%\usepackage{pdfpages}

\title{Matriisilaskentaa}
\author{}
%\email{Ville.Saarikivi@aalto.fi}
%\affil{Otaniemen lukio}
\date{\dmyyyydate\today}

%\renewcommand{\abstractname}{Foreword}
%\setlength{\parskip}{2ex}
%\setlength{\parindent}{0pt}
\usepackage{tasks}
\usepackage{exsheets}


    
%POISTA 2.12.
%\usepackage[parfill]{parskip}    
    

%POISTA 2.12.
%\usepackage[parfill]{parskip}    
\begin{document}
\begin{titlepage}

%\thispagestyle{empty}


\maketitle
\thispagestyle{empty}
\end{titlepage}



%\cleardoublepage
\clearpage

\phantomsection 
\pagenumbering{gobble}
\pdfbookmark{Sisältö}{contents}
%\hypertarget{MyToc}{} 
%\addtocontents{toc}{\protect\hypertarget{toc}{}}
%\addcontentsline{toc}{title}{Contents}

%% Create it. 
%\pagenumbering{arabic}
%\thispagestyle{empty}
%\pagenumbering{arabic}% Arabic page numbers (and reset to 1)

%\renewcommand\contentsname{}
\tableofcontents
%\newpage

%\cleardoublepage
\clearpage
\phantomsection
\pagenumbering{arabic}% Arabic page numbers (and reset to 1)
%%%%%%%%%%%%%%%%%%%%%%%%%%%%%%%%%%%%%%%%%%%%%%%%%%%%%%%%%%%%%%%%%%%%%%%%%%%%%%%%%%%%%%%%%%%%%%%%%%%%%%%%%%%%%%%%%%%%%%%%%%%%%%%%%%%%%%%%%%%%%%%%%%%%%%%%%%%%%%%%%%%%
%%%%%%%%%%%%%%%%%%%%%%%%%%%%%%%%%%%%%%%%%%%%%%%%%%%%%%%%%%%%%%%%%%%%%%%%%%%%%%%%%%%%%%%%%%%%%%%%%%%%%%%%%%%%%%%%%%%%%%%%%%%%%%%%%%%%%%%%%%%%%%%%%%%%%%%%%%%%%%%%%%%%
\section*{Johdanto}
\addcontentsline{toc}{section}{Johdanto}
Tässä monisteessä käsitellään matriisilaskentaa. Moniste on keskeneräinen ja se päivittyy. Jos havaitset virheen, voit kirjoittaa siitä viestin kurssin Classroom-huoneeseen.
%\cleardoublepage
%\clearpage
%\cleardoublepage
\clearpage
\phantomsection
\section*{Merkinnät}
\addcontentsline{toc}{section}{Merkinnät}
\begin{tabular}{ll}
$\mathbf{N}$ & Luonnollisten lukujen $\{0, 1, 2, \ldots\}$ joukko\\
$\mathbf{P}$ & Alkulukujen $\{2, 3, 5, \ldots\}$ joukko\\
$\mathbf{Q}$ & Rationaalilukujen $\{\frac{m}{n}:\, m, n\in\mathbf{Z}\, n\neq 0\}$ joukko\\
$\mathbf{R}$ & Reaalilukujen joukko, $\mathbf{R}=]-\infty, \infty[$\\
$\mathbf{R}_+$ & Positiivisten reaalilukujen joukko, $\mathbf{R}_+=]0, \infty[$\\
$\mathbf{C}$ & Kompleksilukujen $\{z=x+iy:\, x, y\in\mathbf{R}\}$ joukko\\
$\in$ & Kuuluu joukkoon, $a\in \mathbf{N}$ eli alkio $a$ kuuluu luonnollisten lukujen joukkoon\\
$\abs{z}$ & Kompleksiluvun moduli eli itseisarvo\\
$\arg{z}$ & Kompleksiluvun $\neq 0$ argumentti eli vaihekulma, sopimus $\arg{z}\in]-\pi, \pi]$\\
$\vec{x}$ & Vektori, pisteen paikkavektori, sarakevektori\\
$A=(a_{ij})$ & Matriisi, jonka alkiot ovat luvut $a_{ij}$\\
$a$ & Skalaari eli luku\\
$A^T$ & Matriisin transpoosi\\
$A^{\dagger}=A^{\star}$ & Adjungoitu eli hermitoitu matriisi, $A^{\dagger}=(\bar{a_{ji}}),$  eli matriisin $A$ konjugaattitranspoosi. 
\end{tabular}
%\cleardoublepage
\clearpage
\phantomsection
\section{Vektorilaskentaa}
\subsection{Vektorimerkintä matriisilaskennassa}
Vektoreille käytetään tässä monisteessa lihavoituja kirjaimia ja skalaareilla, joita ovat mm. reaali- kompleksiluvut, tavallisia kirjaimia. Toisin sanoen, $\vek{v}$ on vektori mutta $a$ on skalaari.

Samastetaan $n$-ulotteisen avaruuden piste $(x_1, x_2, \ldots, x_n)$ ja sen paikkavektori $\vec{x}=x_1\vec{e}_1+x_2\vec{e}_2+\cdots +x_n\vec{e}_n,$ missä $n$-ulotteisen avaruuden kantavektorit ovat $\vec{e}_1, \vec{e}_2, \ldots, \vec{e}_n,$ missä
$$
\vec{e}_k=(0, 0,\ldots, \overset{k.}{1}, 0, 0, \ldots, 0).
$$
Kolmiulotteisen reaaliavaruuden kantavektorit $\vec{i}, \vec{j},$ ja $\vec{k}$ ovat siis pisteiden $(1,0,0), \ (0,1,0),\ (0,0,1)$ paikkavektorit eli 
$$
\bar{i}=(1,0,0),\quad \bar{j}=(0,1,0)\quad\text{ja}\quad \bar{k}=(0,0,1).
$$
Vektoreille käytetään monesti myös yläviivamerkintää $\bar{i}$, erityisesti käsin kirjoittaessa, kun sekaannuksen vaaraa kompleksikonjugaattiin ei ole.

Matriisilaskentaa ajatellen vektori on mielekästä merkitä pystyvektorina
$$
\vec{x}=\begin{pmatrix}
x_1\\x_2\\ \vdots\\ x_n
\end{pmatrix}
=(x_1, x_2, \ldots, x_n)^T,
$$
missä $\vec{x}^T$ tarkoittaa vektorin tranpoosia (vaakavektorin alkioita ei aina eroteta pilkuilla). Vektoreilla transponointi muuttaa vaakavektorin pystyvektoriksi ja pystyvektektorin vaakavektoriksi.
\begin{esim} Esitä avaruuden $\mathbf{R}^3$ vektori
$\vek{a}=\bar{i}-\pi\bar{j}+12\bar{k}$ pystyvektorina.
\end{esim}
\begin{rat}
Vektorin komponenttiesityksestä saadaan suoraan
$$
\vec{a}=\begin{pmatrix}
1\\
-\pi\\
12
\end{pmatrix}.
$$
\end{rat}

\subsection{Piste- eli skalaaritulo reaalisille vektoreille}
\begin{maar}
$\mathbf{R}^n$:n vektorien $\vec{a}=(a_1, a_2, \ldots, a_n)$ ja $\vec{b}=(b_1, b_2,\ldots, b_n)$ piste- eli skalaaritulo määritellään
$$
\vec{a}\cdot \vec{b}=a_1b_1+a_2b_2+\cdots a_nb_n=\sum_{k=1}^n a_k b_k.
$$
Skalaaritulon avulla voidaan määritellä vektorin pituus
$$
\abs{\vec{a}}=\sqrt{a_1^2+a_2^2+\cdots + a_n^2}=\sqrt{\vec{a}\cdot \vec{a}}$$
ja nollasta poikkeavien vektorien välinen kulma $\alpha$
$$
\vec{a}\cdot \vec{b}=\abs{\vec{a}}\abs{\vec{b}}\cos \alpha.
$$
Pystyvektorimerkinnän avulla pistetulo on lyhyesti vain
$$
\vec{a}\cdot \vec{b}=\vec{a}^T\vec{b}=(a_1, a_2, \ldots, a_n)\begin{pmatrix}
b_1\\
b_2\\
\vdots\\
b_n
\end{pmatrix}
$$
\end{maar}

\subsection{Piste- eli sisätulo kompleksisille vektoreille*}
%\addcontentsline{toc}{subsection}{Piste- eli sisätulo kompleksisille vektoreille}

Kompleksiluvun $z=x+iy,\ x,y \in\mathbf{R},\ i^2=-1,$ itseisarvo eli pituus on sen etäisyys origosta eli $\abs{z}=\sqrt{x^2+y^2}.$ Pätee
$$
\abs{z}=\sqrt{\bar{z}z}, 
$$
missä kompleksiluvun $z$ kompleksikonjugaatti on $\bar{z}=\bar{x+iy}=x-iy=z^{\star}.$
\begin{maar}
Kompleksialkioisten vektorien $\vec{v}=(v_1, v_2, \ldots, v_n)^T$ ja $\vec{u}=(u_1, u_2, \ldots, u_n)^T$, missä $v_k, u_k$ ovat kompleksilukuja, välinen sisätulo on määritellään
$$
\vec{v}\cdot \vec{u}=\sum_{k=1}^n \bar{v_k}u_k=\vec{v}^{\dagger} \vec{u},
$$
missä $\vec{v}^{\dagger}=\bar{\vec{v}}^T$ on vektorin kompleksikonjugaatin transpoosi. Siis
$$
\vec{v}^{\dagger}=\bar{\vec{v}}^T=\bar{\begin{pmatrix}
v_1\\
v_2\\
\ldots\\
v_n
\end{pmatrix}}^T
=\begin{pmatrix}
\bar{v_1}\\
\bar{v_2}\\
\ldots\\
\bar{v_n}
\end{pmatrix}^T
=\left(\bar{v_1}, \bar{v_2}, \ldots, \bar{v_n}\right).
$$
Lisäksi pätee $\vec{v}^{\dagger}=\bar{\vec{v}^T}$\end{maar}

\begin{esim} Laske kompleksivektorien $\vec{a}=(1-i, 3)$ ja $\vec{b}=(-1-i, 2+i)$ sisätulo.
\end{esim}
\begin{rat} Taas tulkitaan vektorit pystyvektoreina. Vektorien sisätulo on
\begin{align*}
\vec{a}\cdot \vec{b}&=\bar{\vec{a}}^T \vec{b}=\bar{(1-i, 3)}\begin{pmatrix}
-1-i\\
2+i
\end{pmatrix}
=(\bar{1-i}, \bar{3})\begin{pmatrix}
-1-i\\
2+i
\end{pmatrix}
=(1+i, 3)\begin{pmatrix}
-1-i\\
2+i
\end{pmatrix}\\
&= (1+i)(-1-i)+3(2+i)=-1-i-i-i^2+6+3i=6+i.
\end{align*}
\end{rat}

\begin{maar}
Vektorien $\vec{u}_i\neq \vec{0}$, $i=1,2,\ldots, n,$ sanotaan olevan ortogonaalisia, jos ne ovat kohtisuorassa toisiaan vastaan eli pistetulo häviää
$$
\vec{u}_i\cdot \vec{u}_j=\vec{u}_i^{\dagger}\vec{u}_j=0,\quad i\neq j.
$$
Jos vektorit $\vec{u}_i$ ovat ykkösen pituisia, 
$$
\abs{\vec{u}_i}=\sqrt{\vec{u}_i\cdot \vec{u}_i}=1,$$
sanotaan niitä normitetuiksi. Jos vektorit $\vec{u}_i$ ovat sekä ortogonaalisia että normitettuja, niin niiden sanotaan olevan ortonormaaleja. Vektorien ortonormaalisuus voidaan ilmaista lyhyesti
$$
\vec{u}_i^{\dagger}\vec{u}_j=\delta_{ij}=\begin{cases}
0,& i\neq j\\
1, & i=j.
\end{cases}
$$
\end{maar}

Käytännön ongelmissa kannattaa käyttää ortonormaaleja kantavektoreja, jolloin laskut helpottuvat. Ongelmaksi muodostuu ongelman kannalta ''mielekkään'' ortonormaalin kannan löytäminen.

\clearpage
\section{Matriisilaskentaa}

\subsection{Matriisin käsite ja perusmääritelmiä}

$m\times n$-matriisi $A$ on lukukaavio, jossa on $m$ riviä ja $n$ saraketta eli pystyriviä. Matriisit kirjoitetaan isolla kirjaimella. Matriisin alkiot ovat useimmiten lukuja, mutta ne voivat olla myös funktioitakin, niitä merkitään pienillä kirjaimilla. Merkitään
$$
A=
\begin{pmatrix} 
   a_{11} &  a_{12}  & \ldots & a_{1n}\\
a_{21}  &  a_{22} & \ldots & a_{2n}\\
\vdots & \vdots & \ddots & \vdots\\
a_{m1}  &   a_{m2}       &\ldots & a_{mn}     
    \end{pmatrix}
$$
tai lyhyesti vain $A\in\mathbf{R}^{m\times n},\ A=(a_{ij}),$ jos matriisin alkiot $a_{ij}$ ovat reaalisia, ja $A\in\mathbf{C}^{m\times n},\ A=(a_{ij}),$ jos alkiot ovat kompleksisia. Koska kompleksiluvut sisältävät reaaliluvut, joten matriisien laskusäännöt ovat voimassa niin reaalisille kuin kompleksisille matriiseille. 

Pystyvektori 
$$\vec{x}=\begin{pmatrix}
x_1\\x_2\\ \vdots\\ x_n\end{pmatrix}$$
on myös matriisi, nimittäin $n\times 1$-matriisi, mutta usein se jätetään merkitsemättä. Matriisilaskennassa $n$-ulotteinen vektori $\vec{x}$ tarkoittaa pystyvektoria.

\begin{maar}
Jos matriisia $A=(a_{ij})$ kerrotaan luvulla $t$, niin saadaan uusi matriisi, jonka kaikki alkiot on kerrottu luvulla $t$ eli $tA=(ta_{ij}).$

Kaksi matriisia $A=(a_{ij})$ ja $B=(b_{kn})$ ovat yhtä suuret, jos ne ovat samankokoiset ja niissä samat alkiot samoille paikoille. Toisin sanoen $A=B$ täsmälleen silloin, kun
\begin{enumerate}[leftmargin=*, topsep=0pt, itemsep=0pt, parsep=0pt]
\item $A$ ja $B$ ovat $m\times n$-matriiseja 
\item $a_{ij}=b_{ij}$ kaikilla $i=1,2,\ldots, m$ ja $j=1,2,\ldots, n.$
\end{enumerate}

Samankokoiset matriisit voi laskea yhteen ja vähentää toisistaan. Summamatriisi lasketaan alkioittain, toisin sanoen, jos $A=(a_{ij})$ ja $B=(b_{ij})$ ovat $m\times n$-matriiseja, niin $C=A+B,$ missä $C$:n alkio $c_{ij}=a_{ij}+b_{ij}$. Summamatriisin koko on sama kuin yhteenlaskettavien matriisien.
\end{maar}

\begin{maar}[Transpoosi]
Jos $m\times n$-matriisin $A$ rivit ja sarakkeet vaihdetaan keskenään eli alkiot peilataan diagonaalin (päälävistäjän suhteen), saadaan matriisin transpoosi $A^T$, joka on kokoa $n\times m$
$$
A^T=\begin{pmatrix} 
   a_{11} &  a_{12}  & \ldots & a_{1n}\\
a_{21}  &  a_{22} & \ldots & a_{2n}\\
\vdots & \vdots & \ddots & \vdots\\
a_{m1}  &   a_{m2}       &\ldots & a_{mn}     
    \end{pmatrix}^T=
    \begin{pmatrix} 
   a_{11} &  a_{21}  & \ldots & a_{m1}\\
a_{12}  &  a_{22} & \ldots & a_{m2}\\
\vdots & \vdots & \ddots & \vdots\\
a_{1n}  &   a_{2n}       &\ldots & a_{mn}     
    \end{pmatrix}.
$$
Jos $A=(a_{ij})$, niin $A^T=(a_{ji}).$
\end{maar}





\begin{maar}[Neliömatriisi]
Jos matriisissa $A$ on yhtä monta riviä kuin saraketta ($m=n$), sitä kutsutaan neliömatriisiksi. Neliömatriisia, jonka päälävistaja- eli diagonaalialkiot ovat ykkösiä ja muut alkiot nollia, kutsutaan identiteettimatriisiksi tai yksikkömatriisiksi
$$
I=\begin{pmatrix}
1 & 0 & \dots & 0\\
0 & 1 & \dots & 0\\
\vdots & \vdots & \ddots & \vdots\\
0 & 0 & \dots & 1\\
\end{pmatrix}.
$$
\end{maar}

\begin{maar}[Symmetrinen ja antisymmetrinen matriisi] Matriisi $A$, joka on itsensä transpoosi, on symmetrinen. Siis $A=A^T$, joten $A$ on neliömatriisi, jonka alkiot sijaitsevat symmetrisesti päälävistäjän suhteen, eli
$$a_{ij}=a_{ji}\quad\text{kaikilla }i \text{ ja }j.$$

Matriisia $A$, jolle pätee $A^T=-A$, sanotaan antisymmetriseksi.
\end{maar}

\begin{maar} Neliömatriisi $D=(d_{ij})$ on diagonaalinen eli diagonaalimatriisi, jos $a_{ij}=0,$ kun $i\neq j.$ Usein merkitään $D=\diag(d_{11}, d_{12},\ldots, d_{nn}).$

Neliömatriisi $L=(l_{ij})$ on alakolmiomatriisi, jos kaikki sen päälävistäjän yläpuoliset alkiot ovat nollia, eli
$b_{ij}=0$, kun $i<j.$

Neliömatriisi $U=(u_{ij})$ on yläkolmiomatriisi, jos kaikki sen päälävistäjän alapuoliset alkiot ovat nollia, eli
$c_{ij}=0$, kun $i>j.$
\end{maar}
\subsection{Perusmääritelmiä kompleksisille matriiseille*}
Adjungointi vastaa reaalisen matriisin transponointia.
\begin{maar}[Adjungaatti]
Jos $m\times n$-matriisin $A$ alkiot konjugoidaan ja rivit ja sarakkeet vaihdetaan keskenään eli alkiot peilataan päädiagonaalin eli päälävistäjän, jonka alkiot ovat $a_{ii},$ suhteen, saadaan matriisin hermitoitu matriisi eli adjungaatti eli hermiittiskonjugaatti $A^{\dagger}=A^{\star}$, joka on kokoa $n\times m$
$$
A^{\dagger}=\bar{\begin{pmatrix} 
   a_{11} &  a_{12}  & \ldots & a_{1n}\\
a_{21}  &  a_{22} & \ldots & a_{2n}\\
\vdots & \vdots & \ddots & \vdots\\
a_{m1}  &   a_{m2}       &\ldots & a_{mn}     
    \end{pmatrix}}^T=
    \begin{pmatrix} 
   \bar{a_{11}} &  \bar{a_{21}}  & \ldots & \bar{a_{m1}}\\
\bar{a_{12}}  &  \bar{a_{22}} & \ldots & \bar{a_{m2}}\\
\vdots & \vdots & \ddots & \vdots\\
\bar{a_{1n} } &   \bar{a_{2n}}       &\ldots & \bar{a_{mn}}   
    \end{pmatrix}.
$$
Jos $A=(a_{ij})$, niin $A^{\dagger}=(\bar{a_{ji}}).$
\end{maar}

Symmetristä matriisia vastaa kompleksinen hermiittinen matriisi.
\begin{maar}[Hermiittinen matriisi]
Jos matriisille $A$ pätee $A^{\dagger}=A$, niin matriiisin sanotaan olevan hermiittinen eli itseadjungoitu. Tällöin $A$ on neliömatriisi, jonka alkioille pätee
$$a_{ij}=\bar{a_{ji}}\quad\text{kaikilla }i \text{ ja }j.$$
Huomaa, että symmetrinen reaalinen matriisi on hermiittinen, joten reaalisten matriisien tapauksessa ei puhutu hermiittisyydestä.

Monissä kvanttimekaniikan ongelmissa esiintyy hermiittisiä operaattoreita, joita kuvataan hermiittisillä matriiseilla.
\end{maar}


\clearpage
\subsection{Matriisin ja vektorin tulo, lineaarinen yhtälöryhmä}
Tarkastellaan $m$:n yhtälön ja $n$:n tuntemattoman lineaarista yhtälöryhmää
$$
\begin{cases}
 \begin{array}{@{}*{7}{c@{}}}
    a_{11} x_1 & {}+{} & a_{12} x_2 & {}+ \cdots +{} & a_{1n} x_n & {}={} & b_1\\
    a_{21} x_1 & {}+{} & a_{22} x_2 & {}+ \cdots +{} & a_{2n} x_n & {}={} & b_2\\[-2pt]
    \vdots     &       & \vdots     &                & \vdots     &       & \vdots\\
    a_{m1} x_1 & {}+{} & a_{m2} x_2 & {}+ \cdots +{} & a_{mn} x_n & {}={} & b_m.
  \end{array}
  \end{cases}$$
Yhtälö kirjoitetaan lyhyesti vain
$$
A\vec{x}=\vec{b},
$$
missä
$$
A=
\begin{pmatrix} 
   a_{11} &  a_{12}  & \ldots & a_{1n}\\
a_{21}  &  a_{22} & \ldots & a_{2n}\\
\vdots & \vdots & \ddots & \vdots\\
a_{m1}  &   a_{m2}       &\ldots & a_{mn}     
    \end{pmatrix},\quad
    \vec{x}=\begin{pmatrix}
    x_1\\
    x_2\\
    \vdots\\
    x_n
    \end{pmatrix},\quad
    \vec{b}=\begin{pmatrix}
    b_1\\
    b_2\\
    \vdots\\
    b_m
    \end{pmatrix}.
$$
Yhtälön ratkaisujen lukumäärä eli erilaisten vektorien $\vec{x}$ määrä voi olla joko $1$, ei yhtään tai äärettömän monta. Jatkossa tutkitaan tapausta, jossa yhtälöitä on yhtä monta kuin tuntemattomia ($m=n$).

Matriisin ja vektorin tulo $A\vec{x}$ on määritelty vain, jos matriisissa $A$ on yhtä monta saraketta kuin vektorissa $\vec{x}$ on rivejä. Kun $A$ on $m\times n$-matriisi ja $\vec{x}$ on $n$-vektori, on niiden tulo $A\vec{x}=\vec{b}$ $m$-vektori, jonka $i$:s alkio $b_i$ saadaan laskemalla matriisin $A$ $i$:nnen rivivektorin ja $\vec{x}$:n reaalinen pistetulo
$$
a_{i1}x_1+a_{i2}x_2+\cdots +a_{in}x_n=\begin{pmatrix}
a_{i1}, & a_{i2}, & \cdots, & a_{in}
\end{pmatrix}
\begin{pmatrix}
x_1\\
x_2\\
\vdots\\
x_n
\end{pmatrix}
=b_i.
$$

Jos matriisi annetaan sarakevektorien avulla
$$
A=(\vec{a}_1, \vec{a}_2, \ldots, \vec{a}_n),\quad \vec{a}_k=\begin{pmatrix}
a_{1k}\\
a_{2k}\\
\vdots\\
a_{mk}
\end{pmatrix},\quad k=1,2, \ldots, n,
$$
niin tulo $A\vec{x}$ on sarakevektorien lineaarikombinaatio
$$
A\vec{x}=x_1\vec{a}_1+x_2\vec{a}_2+\cdots + x_n\vec{a}_n.
$$
\begin{maar}[Lineaarikuvaus matriisin avulla]
Kuvaus $A:\mathbf{R}^n\to\mathbf{R}^m$ on lineaarinen eli se on lineaarikuvaus, jos
$$
A(\alpha \vec{x}+\beta \vec{y})=A(\alpha\vec{x})+A(\beta\vec{y})=\alpha A(\vec{x})+\beta A(\vec{y})
$$
kaikilla vektoreilla $\vec{x}, \vec{y}\in\mathbf{R}^n$ ja kaikilla luvuilla eli skalaareilla $\alpha, \beta\in\mathbf{R}.$ Lisäksi matriisilaskennassa $A(\vec{x})=A\vec{x},$ koska lineaarikuvaus $A:\mathbf{R}^n\to\mathbf{R}^m$ ja sitä vastaava $m\times n$-matriisi $A$ samastetaan. Lineaarikuvauksilla voidaan esittää mm. kiertoja ja peilauksia eri avaruuksien välillä. 
\end{maar}

Mikä sitten määrää lineaarikuvauksen matriisin? Matriisiesitys riippuu siitä, mitä lähtövektoriavaruuden $\mathbf{R}^n$ kantavektoreita käytetään. Matriisiesitystä varten riittää tietää, mitä lineaarikuvaus ''tekee'' kantavektoreilla. Esimerkiksi, jos $\vec{e}_i, i=1, 2, \ldots, n$ ovat kantavektorit, niin riittää laskea
$$
A\vec{e}_i
$$
kaikilla $i=1,2,\ldots, n.$ Tällöin $A$:n matriisiesitykseen kootaan sarakevektoreiksi vektorit $A\vec{e}_i.$


\begin{esim} Lineaarikuvauksesta $A$ tiedetään, että
$$
A(1,2,2)=(2,-4,1),\ A(-1,0,4)=(0, -2, 3)\ \text{ja }  A(-1,-2,-6)=(0, 1, 2).
$$
Määritä $A$:n matriisiesitys.
\end{esim}
\begin{rat}
Pitää siis ratkaista $A\vec{i},\ A\vec{j}$ ja $A\vec{k},$ missä $\vec{i}=\vec{e_1}=(1, 0, 0)^T$ jne. Nyt pystyvektorimerkinnöin saadaan lineaarisuuden perusteella
\begin{align*}
&A\begin{pmatrix}
1\\
2\\
2
\end{pmatrix}=
A\begin{pmatrix}
1\\
0\\
0
\end{pmatrix}
+A\begin{pmatrix}
0\\
2\\
0
\end{pmatrix}+A\begin{pmatrix}
0\\
0\\
2
\end{pmatrix}
=A\vec{i}+2A\vec{j}+2A\vec{k}=\begin{pmatrix*}[r]
2\\
-4\\
1
\end{pmatrix*},\\
&A\begin{pmatrix*}[r]
-1\\
0\\
4
\end{pmatrix*}=
-A\vec{i}+4A\vec{k}=\begin{pmatrix*}[r]
0\\
-2\\
3
\end{pmatrix*},\\
&A\begin{pmatrix*}[r]
-1\\
-2\\
-6
\end{pmatrix*}=
-A\vec{i}-2A\vec{j}-6A\vec{k}=\begin{pmatrix*}[r]
0\\
1\\
2
\end{pmatrix*}.
\end{align*}
Kun nämä kolme yhtälöä lasketaan puolittain yhteen, saadaan
\begin{align*}-A\vec{i}=\begin{pmatrix*}[r]
2\\
-4\\
1
\end{pmatrix*}+\begin{pmatrix*}[r]
0\\
-2\\
3
\end{pmatrix*}+\begin{pmatrix*}[r]
0\\
1\\
2
\end{pmatrix*}=\begin{pmatrix*}[r]
2\\
-5\\
6
\end{pmatrix*}\iff A\vec{i}=\begin{pmatrix*}[r]
-2\\
5\\
-6
\end{pmatrix*}
\end{align*}



Toisesta yhtälöstä saadaan
\begin{align*}
A\vec{k}=\frac{1}{4}\left( \begin{pmatrix*}[r]
0\\
-2\\
3
\end{pmatrix*}+A\vec{i}\right)
=\frac{1}{4}\left( \begin{pmatrix*}[r]
0\\
-2\\
3
\end{pmatrix*}+\begin{pmatrix*}[r]
-2\\
5\\
-6
\end{pmatrix*}\right)=\begin{pmatrix*}[r]
-\frac{1}{2}\\
\frac{3}{4}\\
-\frac{3}{4}
\end{pmatrix*},
\end{align*}
ja edelleen ensimmäisestä yhtälöstä saadaan
\begin{align*}
A\vec{j}
=\frac{1}{2}\left(\begin{pmatrix*}[r]
2\\
-4\\
1
\end{pmatrix*}
-A \vec{i}-2A\vec{k}
\right)
=\frac{1}{2}\left(\begin{pmatrix*}[r]
2\\
-4\\
1
\end{pmatrix*}-\begin{pmatrix*}[r]
-2\\
5\\
-6
\end{pmatrix*}-2\begin{pmatrix*}[r]
-\frac{1}{2}\\
\frac{3}{4}\\
-\frac{3}{4}
\end{pmatrix*} \right)=\begin{pmatrix*}[r]
\frac{5}{2} \\
-\frac{21}{4}\\
\frac{17}{4}
\end{pmatrix*}.
\end{align*}
Lineaarikuvauksen matriisiesitys on
\begin{align*}
A=\left(A\vec{i},\ A\vec{j},\ A\vec{k}\right)=\begin{pmatrix*}[r]
-2 & \frac{5}{2} & -\frac{1}{2} \\
5 & -\frac{21}{4} &\frac{3}{4}\\
-6 & \frac{17}{4} & -\frac{3}{4}	 	
\end{pmatrix*}.
\end{align*}
\end{rat}
\clearpage
\subsection{Matriisien tulo}

Yleistetään matriisin ja vektorin tulo kahden matriisin tuloksi. 

\begin{maar} Olkoot $A$ $m\times n$-matriisi ja $B$ $k\times l$ matriisi. Tulo $AB$ on määritelty, jos ja vain jos $k=n$, toisin sanoen $A$:n sarakkeiden lukumäärä on sama kuin $B$:n rivien. Tällöin tulomatriisi $C=AB$ on $m\times l$-matriisi, jonka alkio $c_{ij}$ saadaan laskettua $A$:n 
$i$:nnen rivivektorin ja $B$ $j$:nnen sarakevektorin (reaalisesta) pistetulosta. Siis
\begin{align*}
AB&=\begin{pmatrix} 
   a_{11} &  a_{12}  & \ldots & a_{1n}\\
a_{21}  &  a_{22} & \ldots & a_{2n}\\
\vdots & \vdots & \ddots & \vdots\\
a_{i1} & a_{i2} & \ldots & a_{in}\\
\vdots & \vdots & \ddots & \vdots\\
a_{m1}  &   a_{m2}       &\ldots & a_{mn}     
    \end{pmatrix}
    \begin{pmatrix} 
   b_{11} &  b_{12}  & \ldots & b_{1j}& \ldots & b_{1l}\\
b_{21}  &  b_{22} & \ldots & b_{2j} & \ldots & b_{2l}\\
\vdots & \vdots & \ddots & \vdots&\cdots&\vdots\\
a_{n1}  &   a_{n2}       &\ldots& b_{nj}& \ldots& a_{nl}     
    \end{pmatrix}
    \\
    &=
    \begin{pmatrix} 
   c_{11} &  c_{12}  & \ldots & c_{1j}&\ldots & c_{1l}\\
c_{21}  &  c_{22} & \ldots & c_{2j}&\ldots &c_{2l}\\
\vdots & \vdots & \ddots & \vdots&\ldots&\vdots\\
c_{i1} & c_{i2} & \ldots & c_{ij}&\ldots & c_{il}\\
\vdots & \vdots & \ddots &\vdots &\ldots& \vdots\\
c_{m1}  &   c_{m2}       &\ldots &c_{mj}&\ldots& c_{ml}     
    \end{pmatrix}=C,
\end{align*}
missä
\begin{align*}
c_{ij}&=\begin{pmatrix}
a_{i1}, & a_{i2}, & \ldots,& a_{in}
\end{pmatrix}
\begin{pmatrix}
b_{1j}\\
b_{2j}\\
\vdots\\
b_{nj}
\end{pmatrix}\\
&=a_{i1}b_{1j}+a_{i2}b_{2j}+\cdots+a_{in}b_{nj}=\sum_{k=1}^n a_{ik}b_{kj}.
\end{align*}

\begin{huom} Matriisien tulo $AB$ ei ole vaihdannainen, eli ei päde, että $AB=BA,$ vaikka matriisit $A$ ja $B$ olisivat samankokoisia neliömatriiseja.

Myöskään tulon nollasääntö ei päde matriiseille. Ehdosta $AB=0,$ jossa $0$ on nollamatriisi, \uuline{ei} seuraa, että $A$ tai $B$ olisi nollamatriisi.
\end{huom}
\end{maar}

\clearpage
\subsection{Matriisin determinantti}
Ainoastaan neliömatriisille $A$ voidaan laskea determinantti, $\det A$, joka on luku. Jos $A$ on $2\times 2$-matriisi
$$
A=\begin{pmatrix}
a_{11} & a_{12}\\
a_{21}& a_{22}
\end{pmatrix},
$$
niin sen determinantti on
$$
\det A=\begin{vmatrix}
a_{11} & a_{12}\\
a_{21}& a_{22}
\end{vmatrix}=a_{11}a_{22}-a_{12}a_{21}.
$$

Neliömatriisien laskenta palautuu $2\times 2$-matriisin determinanttien laskentaan purkusäännöllä. 

\begin{maar}[Determinantin kehityskaava] Neliömatriisin $A=a_{ij}, \ i, j=1, 2, \ldots, n,$ alkiota $a_{ij}$ vastaava alimatriisi	$A_{ij}$ on $(n-1)\times(n-1)$-matriisi, joka on saatu matriisista $A$ poistamalla siitä $i$:s rivi ja $j$:s sarake. Matriisin determinantti voidaan laskea joko kehittämällä se sarakkeen suhteen tai rivin suhteen
$$
\det A=\sum_{i=1}^n (-1)^{i+j} a_{ij}\det A_{ij}=\sum_{j=1}^n (-1)^{i+j} a_{ij}\det A_{ij}.
$$
Kehittämistä jatketaan siihen saakka, kunnes alimatriisit ovat $2\times 2$-matriiseja.
\end{maar}

\begin{lause}
Olkoot $A$ ja $B$ $n\times n$-matriiseja ja $\alpha$ luku. Tällöin pätee
$$
\det(AB)=\det A\det B\quad\text{ja}\quad \det (\alpha A)=\alpha^n\det A.
$$
Lisäksi purkusäännöistä seuraa suoraan, että
$$
\det A^T=\det A.
$$

Jos matriisi $A$ diagonaalinen, yläkolmio- tai alakolmiomatriisi, niin sen determinantti on diagonaalialkioiden tulo
$$
\det A=a_{11}a_{22}\ldots a_{nn}.
$$

\end{lause}

\begin{esim}
Millä parametrin $t$ arvolla $\det A=0,$ kun
$$
A=\begin{pmatrix}
t-1&-1+i\\
-1-i&t+1
\end{pmatrix}.
$$
\end{esim}
\begin{rat}
Lasketaan determinantti
$$
\det A= \begin{vmatrix}
t-1&-1+i\\
-1-i&t+1
\end{vmatrix}=
(t-1)(t+1)-(-1+i)(-1-i)=t^2-1-\abs{-1+i}^2=t^2-1-(1^2+1^2)=t^2-3.
$$
Siis $\det A=0$ täsmälleen silloin, kun $t^2=3$ eli $t=\pm\sqrt{3}.$
\end{rat}

Determinantin purkusäännöistä voidaan johtaa säännöt, joiden avulla determinanttien laskeminen helpottuu.
\begin{lause}[Determinantin ominaisuuksia]
\begin{enumerate}[leftmargin=*, label=\alph*), topsep=3pt, itemsep=9pt, parsep=0pt, font=\bfseries\color{red}]
\item Jos $B$ on saatu matriisista $A$ vaihtamalla sen kaksi riviä keskenään, niin $\det B=-\det A.$
\item Jos $B$ on saatu matriisista $A$ siten, että johonkin $A$:n riviin on lisätty toinen rivi vakiolla kerrottuna, niin $\det B=\det A.$
\item Jos matriisissä $B$ on kaksi samaa riviä, niin $\det B=0.$
\end{enumerate}
\end{lause}

\clearpage
\subsection{Vektorien ristitulo}
Kahden vektorin välinen ristitulo on vektori, joka on kohtisuorassa molempia vektoreita vastaan. Näin ollen sitä ei voida määritellä tason vektoreille, ellei tason $\mathbf{R}^2$ vektoreita ajatella osana avaruuden $\mathbf{R}^3$ vektoreita. Lisäksi ristitulovektorin suunta saadaan oikean käden säännöllä.

\begin{tikzpicture}[thick,scale=2, every node/.style={scale=2}]
\draw[-,fill=white!95!red](0,0)--(3,0)--(4,1)--(1,1)--cycle;
\node at (2,0.5) {$|\textcolor{blue}{\vec{a}}\times \textcolor{red}{\vec{b}}|$};
\draw[ultra thick,-latex,blue](0,0)--(3,0)node[midway,below]{$\vec{a}$};
\draw[ultra thick,-latex,red](0,0)--(1,1)node[midway,above]{$\vec{b}$};
\draw[ultra thick,-latex,blue!50!red](0,0)--(0,3)node[pos=0.7,right]{$\vec{a}\times \vec{b}$};
\draw (0.6,0) arc [start angle=0,end angle=45,radius=0.6]
node[pos=0.7,right]{$\alpha$};
\end{tikzpicture}

\begin{maar}[Ristitulon geometrinen tulkinta] Olkoot $\vec{a}$ ja $\vec{b}$ kolmiulotteisen avaruuden $\mathbf{R}^3$ vektoreita. Tällöin vektorien ristitulo on
$$
\vec{a}\times \vec{b}=\abs{\vec{a}}\abs{\vec{b}}\sin \alpha \vec{e},
$$
missä $\alpha=\angle(\vec{a}, \vec{b})$ on vektorien välinen kulma ja $\vec{e}$ on yksikkövektori, $\abs{\vec{e}}=1$, jonka suunta on sellainen, että vektorit $\vec{a}, \vec{b}, \vec{e}$ muodostavat oikeakätisen kolmikon (peukalo, etusormi, keskisormi).

\end{maar}
\begin{seur}
Olkoot $\vec{a}, \vec{b}\neq \vec{0}$ avaruuden vektoreita.
\begin{listab} \item Tällöin $\vec{a}\times \vec{b}=\vec{0}$ jos ja vain jos $\vec{a}\parallel \vec{b}$ eli $\vec{a}=t\vec{b}$ jollakin $t\neq 0.$
\item Lisäksi ristitulovektori $\vec{a}\times\vec{b}$ on kohtisuorassa molempia vektoreita $\vec{a}$ ja $\vec{b}$ vastaan eli
$$
(\vec{a}\times\vec{b})\bullet \vec{a}=0\quad\text{ja}\quad (\vec{a}\times\vec{b})\bullet \vec{b}=0,
$$ 
tässä $\vec{v}\bullet\vec{u}$ tarkoittaa vektorien välistä pistetuloa.
\item $\abs{\vec{a}\times\vec{b}}$ on vektorien $\vec{a}$ ja $\vec{b}$ virittämän suunnikkaan pinta-ala. Erityisesti, jos kolmion kaksi sivuvektoria ovat $\vec{a}$ ja $\vec{b},$ niin kolmion pinta-ala on $\dfrac{1}{2}\abs{\vec{a}\times\vec{b}}.$
\end{listab}
\end{seur}
\begin{lause}[Ristitulon komponenttiesitys] Olkoot 
$$
\vec{a}=a_1\vec{i}+a_2\vec{j}+a_3\vec{k}\quad\text{ja}\quad \vec{b}=b_1\vec{i}+b_2\vec{j}+b_3\vec{k}.
$$
Tällöin
$$
\vec{a}\times\vec{b}=(a_2b_3-a_3b_2)\vec{i}+(a_3b_1-a_1b_3)\vec{j}+(a_1b_2-a_2b_1)\vec{k}.
$$
Laskinohjelmistoissa ristitulon voi myös kätevästi laskea suoraan.
\end{lause}
\begin{tod} Harjoitustehtävänä.
\end{tod}
\begin{maar}[Laskukaava $2\times 2$-determinantille]
$2\times 2$-matriisi $A$ on lukukaavio, jossa on 2 riviä ja 2 saraketta,
$$
A=\begin{pmatrix}
a & b\\
c & d
\end{pmatrix}.
$$
$A$:n  determinantti on luku
$$
\det{A}=\begin{vmatrix}
a & b\\
c & d
\end{vmatrix}=ad-bc.
$$
\end{maar}

\begin{lause}[Ristitulo determinanttina] Vektorien $$
\vec{a}=a_1\vec{i}+a_2\vec{j}+a_3\vec{k}\quad\text{ja}\quad \vec{b}=b_1\vec{i}+b_2\vec{j}+b_3\vec{k}.
$$ ristitulo voidaan laskea $3\times3$-determinanttina seuraavan purkusäännön avulla
$$
\vec{a}\times\vec{b}=\begin{vmatrix}
\vec{i}& \vec{j}&\vec{k}\\
a_1 & a_2 & a_3\\
b_1 & b_2 & b_3
\end{vmatrix}
=\begin{vmatrix}
a_2 & a_3\\
b_2 & b_3
\end{vmatrix}\vec{i}-
\begin{vmatrix}
a_1& a_3\\
b_1 & b_3
\end{vmatrix}\vec{j}+
\begin{vmatrix}
a_1 & a_2\\
b_1 & b_2
\end{vmatrix}\vec{k}.
$$
Huomaa, että vektoria $\bar{i}, \bar{j}$ tai $\bar{k}$ vastaava $2\times2$-determinantti saadaan näkyviin poistamalla se rivi ja sarake, jossa vektori on. Huomaa myös kiertovaihtelu $+, -, +$
\end{lause}
\begin{lause}[Ristitulon ominaisuuksia]
Olkoot $\vec{a}, \vec{b}$ ja $\vec{c}$ mielivaltaisia avaruuden vektoreita ja $t$ on reaaliluku. Tällöin on voimassa
\begin{listaa}
\item $\vec{a}\times\vec{a}=\vec{0}$

\item $\vec{a}\times\vec{b}=-\vec{b}\times \vec{a}$ (ristitulo on antikommutatiivinen)

\item $(\vec{a}+\vec{b})\times\vec{c}=\vec{a}\times \vec{c}+\vec{b}\times\vec{c}$ (osittelulaki)



\item $\vec{a}\times (\vec{b}+\vec{c})=\vec{a}\times\vec{b}+\vec{a}\times \vec{c}$ (osittelulaki)

\item $(t\vec{a})\times\vec{b}=\vec{a}\times(t\vec{b})=t(\vec{a}\times \vec{b})$
\item $\vec{a}\bullet(\vec{a}\times\vec{b})=0=\vec{b}\bullet(\vec{a}\times\vec{b})$
\item $\vec{i}\times\vec{i}=\vec{j}\times\vec{j}=\vec{k}\times\vec{k}=\vec{0}$
\item $\vec{i}\times \vec{j}=\vec{k},\ \vec{j}\times \vec{k}=\vec{i}$ ja $\vec{k}\times\vec{i}=\vec{j}.$
\end{listaa}
\end{lause}
\begin{tod} Suoraviivainen lasku.
\end{tod}

\begin{tikzpicture}[rotate around y=-15, rotate around z=7]
          % Nota: els punts tenen coordenades (x,z,y)
          \coordinate (O) at (0,0,0);
          \coordinate (P0) at (3,2,2);
          \coordinate (P) at (5.5,2,4);

          % Punts M i N
          \coordinate (M) at (5.5,2,2);
          \coordinate (N) at (3,2,4);

          % Punts dels vectors directors

          \coordinate (V1) at (4.5,2,2);
          \coordinate (V2) at (3,2,3.5);

          % Punts del pla (a partir de A, P, M i N
          \coordinate (PLA0) at (1,2,0.2);
          \coordinate (PLA1) at (7,2,0.2);
          \coordinate (PLA2) at (1,2,4.5);
          \coordinate (PLA3) at (7,2,4.5);          

          % eixos de coordenades
          \draw[->] (0,0,0) -- (4,0,0);
          \draw[->] (0,0,0) -- (0,3,0);
          \draw[->] (0,0,0) -- (0,0,3);
          \draw (4,0,0) node[anchor=west] {$y$};
          \draw (0,3,0) node[anchor=south] {$z$};
          \draw (0,0,3) node[anchor=north east] {$x$};

          % Pla
          \fill[color=green!10,thick,draw=black] (PLA0) -- (PLA1) -- (PLA3) -- (PLA2) -- cycle;
          \draw (PLA1) node[anchor=west] {$T$};

          % Punts: A, P i vectors de posició i AP
          \draw[color=orange,very thick,->,dashed] (O) -- (P0);
          \draw[color=green,very thick,->,dashed] (O) -- (P);
          \draw[color=red,ultra thick,->] (P0) -- (P);
          \draw (P0) node[anchor=south east] {$A$};
          \draw (P) node[anchor=south] {$P$};

          % Llei del paral·lelogram
          \draw[thick, dotted,->] (P0) -- (M);
          \draw[thick, dotted,->] (P0) -- (N);
          \draw[thick, dotted,->] (M) -- (P);
          \draw[thick, dotted,->] (N) -- (P);         


          % Punts M i N i els respectius vectors

          \draw[ultra thick,color=black,->] (P0) -- (V1) node[midway, above] {$\bar{u}$};
          \draw[ultra thick,color=black,->] (P0) -- (V2) node[midway, left=9pt] {$\bar{v}$};
          \draw (M) node[anchor=west] {$M$};
          \draw (N) node[anchor=east] {$N$};

		\draw[ultra thick,color=black,->] (P0) -- (3,4,2) node[midway, right] {$\bar{n}$};

    \end{tikzpicture}
    
Taso $T$ kulkee pisteen $A$ kautta ja tason suuntavektorit ovat $\vec{u}$ ja $\vec{v}$, joille $\vec{u}\nparallel \vec{v}.$ Taso muodostuu pisteistä $P$, joille
$$\bar{OP}=\bar{OA}+s\vec{u}+t\vec{v},\quad s, t\in \mathbf{R}.$$
Tätä esitystä kutsutaan tason vektoriesitykseksi. Vektorit $\vec{u}$ ja $\vec{v}$ virittävät tason. Miten tästä esityksestä päästään tason yhtälöön eli tason normaalimuotoon $ax+by+cz+d=0$?
Ristitulo antaa tähän suoran vastauksen.
\begin{lause} Jos taso kulkee pisteen $A=(x_0, y_0, z_0)$ kautta ja sen virittävät vektorit $\bar{u}$ ja $\bar{v},$ niin tason normaaliksi sopii vektori
$$\vec{n}=\vec{u}\times \vec{v}$$
tai mikä hyvänsä tämän kanssa yhdensuuntainen vektori. Tason yhtälö on tällöin
$$
\vec{n}\bullet (\vec{r}-\vec{r}_0)=0,
$$
missä 
$$\vec{r}=\vec{OP}=x\vec{i}+y\vec{j}+z\vec{k}\quad\text{ja}\quad \vec{r}_0=\bar{OA}=x_0\vec{i}+y_0\vec{j}+z_0\vec{k}.
$$
\end{lause}

\clearpage
\subsection{Käänteismatriisi}
\begin{maar}
Neliömatriisilla $A$ on käänteismatriisi $B$, jos
$$
AB=BA=I,
$$
jolloin merkitään $B=A^{-1}.$ 
\end{maar}

Itse asiassa ehdosta $AB=I$, seuraa, että $B=A^{-1}$, vaikkei matriisien tulo olekaan vaihdannainen, toisin sanoen  $AB\neq BA.$ Koska tulon determinantti on determinanttien tulo, niin
$$
\det A\det (A^{-1})=\det(AA^{-1})=\det I=1.
$$
Siispä $\det (A^{-1})=\dfrac{1}{\det A}.$ Pätee: Matriisilla $A$ on käänteismatriisi täsmälleen silloin, kun $\det A\neq 0.$

\begin{lause}
Erikoisesti, jos
$$
A=\begin{pmatrix}
a_{11}& a_{12}\\
a_{21} & a_{22}
\end{pmatrix}
$$
on säännöllinen, niin
$$
A^{-1}=\frac{1}{\det A}\begin{pmatrix}
a_{22}& -a_{12}\\
-a_{21} & a_{11}
\end{pmatrix}.
$$
\end{lause}


Tarkastellaan nyt yhtälöryhmää, jossa on yhtä monta tuntematonta kuin yhtälöäkin
$$
\begin{cases}
 \begin{array}{@{}*{7}{c@{}}}
    a_{11} x_1 & {}+{} & a_{12} x_2 & {}+ \cdots +{} & a_{1n} x_n & {}={} & b_1\\
    a_{21} x_1 & {}+{} & a_{22} x_2 & {}+ \cdots +{} & a_{2n} x_n & {}={} & b_2\\[-2pt]
    \vdots     &       & \vdots     &                & \vdots     &       & \vdots\\
    a_{n1} x_1 & {}+{} & a_{n2} x_2 & {}+ \cdots +{} & a_{nn} x_n & {}={} & b_n.
  \end{array}
  \end{cases}$$
eli lyhyesti vain $A\vec{x}=\vec{b}.$  

Yhtälöllä on  $A\vec{x}=\vec{b}$ on yksikäsitteinen ratkaisu täsmälleen silloin, kun $A^{-1}$ on olemassa eli $\det A\neq 0.$ Tällöin ratkaisu on 
$$
\vec{x}=A^{-1}\vec{b}.
$$

Käytännössä -- ilman tietokonetta -- käänteismatriisia ei lasketa, vaan yhtälö $A\vec{x}=\vec{b}$ ratkaistaan Gaussin eliminointimenetelmällä myös niissä tapauksissa, joissa $A$ ei ole neliömatriisi. Gaussin eliminointimenetelmä on algoritmi, joka paljastaa samalla ratkaisujen lukumäärän.

\begin{lause} Olkoot $A$ ja $B$ säännöllisiä samankokoisia neliömatriiseja. Tällöin pätee
$$
\left(AB\right)^{-1}=B^{-1}A^{-1}.
$$
\end{lause}


\subsection{Lineaarinen yhtälöryhmä}
Lineaarinen yhtälöryhmä on muotoa
$$
\begin{cases}
 \begin{array}{@{}*{7}{c@{}}}
    a_{11} x_1 & {}+{} & a_{12} x_2 & {}+ \cdots +{} & a_{1n} x_n & {}={} & b_1\\
    a_{21} x_1 & {}+{} & a_{22} x_2 & {}+ \cdots +{} & a_{2n} x_n & {}={} & b_2\\[-2pt]
    \vdots     &       & \vdots     &                & \vdots     &       & \vdots\\
    a_{m1} x_1 & {}+{} & a_{m2} x_2 & {}+ \cdots +{} & a_{mn} x_n & {}={} & b_m.
  \end{array}
  \end{cases}$$
Matriisilaskennassa lineaarinen yhtälöryhmä kirjoitetaan muodossa $A\vec{x}=\vec{b}.$ Lineaarisella yhtälöryhmällä ratkaisujen lukumäärä, eli erilaisten vektorien $\vec{x}$ määrä, ei voi olla mitä vain.

\begin{lause} Lineaarisella yhtälöryhmällä $A\vec{x}=\vec{b}$ on joko
\begin{listab}
\item täsmälleen yksi ratkaisu $\vec{x}$
\item ei yhtään ratkaisua
\item äärettömän monta ratkaisua.
\end{listab}

\end{lause}

Jos $A$ on neliömatriisi, niin yhtälöllä $A\vec{x}=\vec{b}$ on yksikäsitteinen ratkaisu täsmälleen silloin, kun $A$ on säännöllinen eli matriisilla $A$ on käänteismatriisi. Tällöin ratkaisu on
$$
\vec{x}=A^{-1}\vec{b}.
$$
Käytännössä käänteismatriisi ei kuitenkaan määritetä, vaan käytetään Gaussin algoritmia. Algoritmi nimittäin paljastaa, onko yhtälöllä yksikäsitteistä ratkaisua.

\clearpage

\subsection{Permutaatiomatriisi}
Tarkastellaan tässä neliömatriisia $A.$ Eräs rivioperaatio on rivien vaihtaminen keskenään. Miten tämä saadaan aikaiseksi matriisikertolaskulla?

\begin{maar} Matriiisi $P$ on permutaatiomatriisi, jos se on saatu identiteettimatriisista $I$ vaihtamalla siinä rivien paikkoja.
\end{maar}

\begin{esim} Tarkastellaan matriisin rivien vaihtoa
$$
A=\begin{gmatrix}[p]
1 & 2 & 3 \\
4 & 5 & 6 \\
7 & 8 & 9 
\rowops
\swap{1}{2}
\end{gmatrix}\sim
\begin{pmatrix}
1 & 2 & 3 \\
7 & 8 & 9\\
4 & 5 & 6 
\end{pmatrix}=\tilde{A}.
$$
Pitäisi siis löytää rivienvaihtomatriisi $P$ niin, että $PA=\tilde{A}.$

Vaihdetaan identiteettimatriisista $I$ toisen ja kolmannen rivin paikkoja keskenään, jolloin saadaan matriisi
$$
P=\begin{pmatrix}
1 & 0 &0\\
0 & 0 & 1\\
0 & 1 & 0
\end{pmatrix}.
$$
Kun matriisilla $P$ kerrotaan vasemmalta matriisia $A,$ saadaan matriisi, jonka ylin rivi on sama kuin matriisissa $A,$ mutta kaksi muuta riviä vaihtaa paikkaansa
$$
PA=\begin{pmatrix}
1 & 0 &0\\
0 & 0 & 1\\
0 & 1 & 0
\end{pmatrix}\begin{pmatrix}
1 & 2 & 3 \\
4 & 5 & 6 \\
7 & 8 & 9 
\end{pmatrix}
=
\begin{pmatrix}
1 & 2 & 3\\
7 & 8 &9\\
4 & 5 & 6
\end{pmatrix}=\tilde{A}.
$$
\end{esim}
\clearpage
\subsection{Ominaisarvo ja ominaisvektori}

$n$-ulotteista vektoria $\vec{v}\neq 0$ sanotaan lineaarikuvauksen $A:\mathbf{C}^n\to\mathbf{C}^n$ (eli $n\times n$-matriisin) \emph{ominaisvektoriksi}, jos 
$$
A\vec{v}=\lambda\vec{v}
$$
jollakin luvulla $\lambda.$ Lukua $\lambda$ sanotaan matriisin ominaisarvoksi, ja se voi reaalisenkin matriisin tapauksessa olla kompleksiluku. Ominaisarvo voi myös olla $0,$ jolloin matriisi on epäsäännöllinen.

\begin{tikzpicture}
\draw [-latex, red, thick] (0,0) -- (2,2);
\node[left] at (1.3,1.2) {$\vec{v}$};
\draw [-latex, red, thick] (5,2) -- (2,-1);
\node[right] at (3.5,.5) {$A\vec{v}$};
\draw[->] (1.1,1) to [bend left=45] (2.7,0);
\end{tikzpicture}


Ominaisvektori $\vec{v}$ ei ole yksikäsitteinen, ainoastaan sen suunta on, sillä $A(t\vec{v})=tA\vec{v}=t\lambda \vec{v}=\lambda t\vec{v}.$

Miten määrätään neliömatriisin ominaisarvot ja vektorit? Edeltä muistetaan, että matriisi on säännöllinen -- eli sillä on käänteismatriisi -- täsmälleen silloin, kun $\det A\neq 0.$

Palataan ominaisvektoriyhtälöön $A\vec{v}=\lambda\vec{v}.$ Tästä saadaan yhtäpitävästi
\begin{align*}
A\vec{x}-\lambda\vec{v}&=0\\
\left(A-\lambda I\right)\vec{v}&=0.
\end{align*}
Ominaisvektori ei voi olla nollavektori, joten neliömatriisin $A$ ominaisarvot saadaan yhtälöstä
$$\det\left(A-\lambda I\right)=0,$$
sillä yhtälön $\left(A-\lambda I\right)\vec{v}=0$ kerroinmatriisin pitää olla epäsäännöllinen.

Lauseke $\det\left(A-\lambda I\right)$ on $\lambda$:n suhteen polynomi astetta $n$, jonka juuret ovat ominaisarvot. Lauseketta
$$
p_A(\lambda)=\det\left(A-\lambda I\right)$$
kutsutaankin matriisin $A$ karakteristiseksi polynomiksi. Huomaa, että reaalisen matriisin ominaisarvot voivat olla kompleksilukuja.




Ominaisarvoa ja sitä vastaavaa ominaisvektoria kutsutaan ominaispariksi. Ominaisarvot ja -vektorit lasketaan seuraavasti.
\begin{enumerate}[leftmargin=*]
\item Ominaisarvot saadaan yhtälöstä $p_A(\lambda)=\det\left(A-\lambda I\right)=0.$
\item Ominaisarvoa $\lambda_i$ vastaava ominaisvektori $\vec{v}_i$ ratkaistaan yhtälöstä 
$$
\left(A-\lambda_i I\right)\vec{v}_i=0
$$
esimerkiksi ratkaisemalla yhtälöryhmä komponenttiesityksestä tai Gaussin eliminointimenetelmällä.
\end{enumerate}
Käytännössä ominaisarvot ja -vektorit ratkaistaan hienostuneilla algoritmeilla tietokoneella.

Ominaisarvojen joukkoa kutsutaan $A$:n spektriksi ja vastaavia ominaisvektoreita usein $A$:n ominaistiloiksi.

\begin{maar}[Ominaisarvon algebrallinen kertaluku] Jos matriisin $A$ karakteristisen polynomin $p_A$ juuren $\lambda_i$ asteluku on $n$, eli karakteristisen polynomi on muotoa
$$
p_A(\lambda)=(\lambda-\lambda_i)^n \tilde{p}(\lambda),
$$
niin sanotaan, että ominaisarvon $\lambda_i$ algebrallinen kertaluku on $n$. Tätä merkitään
$$
m_a(\lambda_i)=n.
$$
\end{maar}

Eri ominaisarvoihin liittyvät ominaisvektorit ovat lineaarisesti riippumattomia. Symmetriset matriisit ovat monessa yhteydessä tärkeitä.

\begin{lause} Symmetrisen matriisin $A=A^T$ ominaisarvot ovat reaalisia ja ominaisvektorit ovat ortogonaalisia. Ominaisvektorit $\vec{u}_i$ ovat näin ollen toisiaan vastaan kohtisuorassa
$$
\vec{u}_i\cdot \vec{u}_j=\vec{u}_i^T\vec{u}_j=0,\quad\text{kun}\quad i\neq j.
$$
\end{lause}

\subsection{Neliömuoto}
Olkoon $A$ $n\times n$-matriisi. Tällöin sitä vastaava neliömuoto on
$$
q(\vec{x})=\vec{x}^T A\vec{x}=\sum_{i=1}^n\sum_{j=1}^n a_ij x_ix_j.
$$
Koska $x_ix_j=x_jx_i,$ niin neliömuotoa vastaavaksi matriisiksi valitaan symmetrinen matriisi.

Symmetrinen matriisi $A$ on ortogonaalisesti diagonalisoituva, niin muuttujan vaihdolla neliömuoto voidaan esittää ilman ristitermejä $x_ix_j,\ i\neq j.$


\subsection{Toisen asteen käyrät}
Toisen asteen käyrä on muotoa
$$
f(\vec{x})=a_{11}x^2+a_{22}y^2+2a_{12}xy+b_1x+b_2y+c=0,
$$
lyhyesti vain
$$
f(\vec{x})=\vec{x}^TA\vec{x}+B\vec{x}+c=0,
$$
missä
$$
A=\begin{pmatrix}
a_{11} & a_{12}\\
a_{21} & a_{22}
\end{pmatrix},
a_{21}=a_{12},\quad B=[b_1, b_2],\quad c\in\mathbf{R}.
$$
\clearpage
\subsection{Bra-ket-merkintätapa*}
Bra-ket-merkintätapa eli Diracin merkintätapa on yleisessä käytössä kvanttimekaniikassa. Englanniksi merkintätavan nimi on \emph{bracket} (sulku). Bra-vektoria merkitään symbolilla
$\bra{\psi}$, kun taas ket-vektoria merkitään $\ket{\psi}.$ Ket-vektorilla tarkoitetaan kvanttimekaanista tilaa. Bra-ket-merkintä
$$
\bracket{\psi}{\phi}
$$
tarkoittaa tilojen $\ket{\psi}$ ja $\ket{\phi}$ välistä sisätuloa.

Mikäli tila-avaruus $n$-ulotteinen kompleksiavaruus, niin ket-vektorit $\ket{\vec{a}}$ ja $\ket{\vec{b}}$ esittävät pisteiden paikkavektoreita $n$-ulotteisessa kompleksiavaruudessa
$$
\ket{\vec{a}}=\begin{pmatrix}
a_1\\
a_2\\
\vdots\\
a_n
\end{pmatrix},\quad \ket{\vec{b}}=\begin{pmatrix}
b_1\\
b_2\\
\vdots\\
b_n
\end{pmatrix}.
$$
Ket-vektoria $\ket{\vec{a}}$ vastaava bra-vektori on ket-vektorin adjungaatti eli
$$
\bra{\vec{a}} =\begin{pmatrix}
a_1\\
a_2\\
\vdots\\
a_n
\end{pmatrix}^{\dagger}
=\begin{pmatrix}
\bar{a_1}, \bar{a_2}, \ldots, \bar{a_n}
\end{pmatrix}
$$
Täten vektorien eli tilojen $\ket{\vec{a}}$ ja $\ket{\vec{b}}$ välinen sisätulo 
$$
\bracket{a}{b}=\begin{pmatrix}
\bar{a_1}, \bar{a_2}, \ldots, \bar{a_n}
\end{pmatrix}\begin{pmatrix}
b_1\\
b_2\\
\vdots\\
b_n
\end{pmatrix}=\sum_{j=1}^n\bar{a_j}b_j.
$$

Tilaa $\ket{\psi}$ sanotaan normitetuksi, jos
$$
\norm{\psi}=\sqrt{\bracket{\psi}{\psi}}=1.
$$

Huomaa, että bra-ket-merkintätapa on lineaarinen ket-vektorin suhteen ja konjugaattilineaarinen bra-vektorin suhteen. Olkoon $\alpha, \beta\in \mathbf{C},$ ja olkoon $\bra{\vec{f}}, \bra{\vec{g}}, \bra{\vec{h}}$ vektoreita, tällöin
\begin{enumerate}[leftmargin=*, label=\alph*), topsep=0pt, itemsep=9pt, parsep=0pt, font=\small\bfseries\color{blue}]
\item $\bracket{\vec{f}}{\alpha\vec{g}+\beta \vec{h}}=\bracket{\vec{f}}{\alpha\vec{g}}+\bracket{\vec{f}}{\beta \vec{h}}=\alpha\bracket{\vec{f}}{\vec{g}}+\beta\bracket{\vec{f}}{\vec{h}}$
\item $\bracket{\alpha\vec{f}+\beta \vec{g}}{\vec{h}}=\bracket{\alpha\vec{f}}{\vec{h}}+\bracket{\beta\vec{g}}{ \vec{h}}=\bar{\alpha}\bracket{\vec{f}}{\vec{h}}+\bar{\beta}\bracket{\vec{g}}{\vec{h}}$.
\end{enumerate}

Kun operaattori $A$ (eli matriisi) operoi ket-vektoriin $\ket{\phi}$, saadaan ket-vektori
$$
A\ket{\phi}=\ket{A\phi}.
$$
Vastaavalla tavalla saadaan bra-vektori, kun operaattori $A$ operoi bra-vektoriin $\bra{\psi}$ oikealta
$$
\bra{\psi}A=\bra{\psi A}.
$$

Hermiittisen operaattorin $A$ odotusarvo normitetussa tilassa $\psi$, $\norm{\psi}=1$, voidaan laskea lausekkeesta
$$
\ipr{A}=\matrixel{\psi}{A}{\psi}.
$$

Kvanttimekaniikassa mitattavia fysikaalisia suureita vastaavat hermiittiset operaattorit, joita kuvataan matriiseilla. Systeemin ollessa normitetussa tilassa $\bra{\psi}$,  on mitatun suureen keskiarvo sama kuin suuretta vastaavan operaattorin $A$ odotusarvo. 

\begin{esim} Laske Paulin spinmatriisin
$$
\sigma_y=\begin{pmatrix}
0 & -i\\
i & 0
\end{pmatrix}
$$
odotusarvo tilassa $\displaystyle \ket{\psi}=\frac{1}{\sqrt{2}}\begin{pmatrix}
1\\
-1
\end{pmatrix}.$
\end{esim}
\begin{rat} Tila on normitettu, sillä
$$
\norm{\psi}^2=\bracket{\psi}{\psi}=\frac{1}{\sqrt{2}}\begin{pmatrix}
1 & -1
\end{pmatrix}\frac{1}{\sqrt{2}}\begin{pmatrix}
1\\
-1
\end{pmatrix}=\frac{1}{2}\left(1^2+(-1)^2\right)=1.
$$
Näin ollen Paulin spinmatriisin odotusarvo on
\begin{align*}
\ave{\sigma_y}&=\matrixel{\psi}{\sigma_y}{\psi}=\bracket{\psi}{\sigma_y\psi}=\frac{1}{\sqrt{2}}\begin{pmatrix}
1&
-1
\end{pmatrix}
\begin{pmatrix}
0 & -i\\
i & 0
\end{pmatrix}\frac{1}{\sqrt{2}}\begin{pmatrix}
1\\
-1
\end{pmatrix}\\
&=
\frac{1}{2}\begin{pmatrix}
1&
-1
\end{pmatrix}
\begin{pmatrix}
0 & -i\\
i & 0
\end{pmatrix}\begin{pmatrix}
1\\
-1
\end{pmatrix}
=
\frac{1}{2}\begin{pmatrix}
1&
-1
\end{pmatrix}
\begin{pmatrix}
0\cdot1-i\cdot(-1) \\
i\cdot 1+0\cdot (-1)
\end{pmatrix}\\
&=
\frac{1}{2}\begin{pmatrix}
1&
-1
\end{pmatrix}
\begin{pmatrix}
i \\
i
\end{pmatrix}
=\frac{1}{2}\left(i-i\right)=0.
\end{align*}

\end{rat}
\clearpage
\subsection{Unitaarinen matriisi*}
\begin{maar} Neliömatriisi $U$ on unitaarinen, jos
$$
U^{\dagger}U=UU^{\dagger}=I.
$$
Toisin sanoen matriisin $U$ käänteismatriisi on sen adjungaatti, $U^{-1}=U^{\dagger}.$
\end{maar}

Suora seuraus unitaarisuuden määritelmästä on, että matriisin $U$ sarakevektorit ovat ortonormaaleja.

\begin{lause} Olkoon $U$ unitaarinen $n\times n$-matriisi ja vektorit $\vec{u_i}, \ i=1,2,\ldots, n,$ sen sarakkeet, eli
$$
U=\begin{pmatrix}
\vec{u_1},& \vec{u_2},&\ldots , & \vec{u_n}
\end{pmatrix}.$$
Tällöin vektorit $\vec{u_i}$ ortonormaalit
$$
\bracket{\vec{u_i}}{\vec{u_j}}=\vec{u_i}^{\dagger}\vec{u_j}=\delta_{ij}=\begin{cases}
1,&\text{kun } i=j\\
0,&\text{muuten}.\end{cases}
$$
Vektorien $\vec{u_i},\ i=1,2,\ldots,n,$ ortonormaalisuus tarkoittaa, että vektorit ovat pareittain kohtisuorassa (ortogonaalisia) ja ykkösen pituiset.
\end{lause}

Vektorin kuvaaminen unitaarisella matriisilla säilyttää sisätulon eli vektorien väliset kulmat säilyvät unitaarisissa kuvauksissa.

\begin{lause} Olkoot $U$ unitaarinen $n\times n$-matriisi ja $\vec{x}, \vec{y}\in\mathbf{C}^n.$ Tällöin
$$
\bracket{U\vec{x}}{U\vec{y}}=\bracket{\vec{x}}{\vec{y}}.
$$
\end{lause}
\begin{tod} Suoralla laskulla saadaan
\begin{align*}
\bracket{U\vec{x}}{U\vec{y}}&=\left(U\vec{x}\right)^{\dagger} U\vec{y}=\vec{x}^{\dagger}U^{\dagger}U \vec{y}\\
&=\vec{x}^{\dagger} I \vec{y}=\vec{x}^{\dagger}\vec{y}=\bracket{\vec{x}}{\vec{y}}.
\end{align*}
\end{tod}


Kvanttimekaniikassa operaattoria vastaa hermiittinen matriisi. Hermiittiselle matriisille $A=A^{\dagger}$ pätee:
\begin{enumerate}[leftmargin=*, label=--, topsep=0pt, itemsep=9pt, parsep=0pt, font=\small\bfseries\color{blue}]
\item kaikki ominaisarvot $\lambda_i$ ovat reaalisia
\item eri ominaisarvoja vastaavat ominaisvektorit $\vec{v_i}$ ovat ortogonaaliset eli $\vec{v_i}\cdot \vec{v_j}=0, i\neq j.$
\end{enumerate}
\clearpage
 \section*{Tehtäviä}
\addcontentsline{toc}{section}{Tehtäviä}
\linespread{1.667}
\begin{teht} Olkoon $\bar{u}=(7, 2, 5)^T,\ \vec{v}=(3, 1, 3)^T$ ja $\bar{w}=(6, 1, 0)^T.$ Näytä, että
$3\bar{u}-5\bar{v}=\bar{w}.$ Määritä yhtälön
$$
\begin{pmatrix}
7 & 3\\
2 & 1\\
5 & 3
\end{pmatrix} \vec{x}=\begin{pmatrix}
6\\1\\0
\end{pmatrix}
$$
ratkaisu.
\end{teht}

\begin{teht} Määritä tason lineaarikuvaus eli matriisi, joka peilaa pisteen $\vec{x}\in\mathbf{R}^2$
\begin{listaa}
\item $x_1$-akselin suhteen
\item $x_2$-akselin suhteen
\item suoran $x_1+x_2=0$ suhteen
\item origon suhteen?
\end{listaa}
\end{teht}
\begin{teht}
Määritä edellisen tehtävän perusteella matriisi, joka ensin peilaa pisteen $\vec{x}\in\mathbf{R}^2$ suoran $x_1+x_2=0$ suhteen ja sitten peilaa pisteen origon suhteen? Onko matriisi sama, jos peilausjärjestystä vaihdetaan?
\end{teht}


\begin{teht} Olkoon $A=\begin{pmatrix}
1 & 2\\
1 & 3\\
1 & 5\\
\end{pmatrix}.$ Määrää $2\times 3$-matriisi $B,$ jonka alkioina ovat 1, -1 ja 0, niin, että $AB=I_2.$ Onko mahdollista, että $CA=I_4$ jollakin $4\times 2$-matriisilla $C$? Tässä $I_n$ on $n\times n$-yksikkömatriisi.
\end{teht}

\begin{teht}
Neliömatriisi $N$ on nilpotentti, jos
$
N^k=\overbrace{N\cdot N \cdots N}^{k\text{ kpl}}=0.
$
Anna esimerkki sellaisesta matriisista $N$, jonka kaikki alkiot poikkeavat nollasta, mutta jolle pätee $N^2=0.$
\end{teht}

\begin{teht} Olkoon $A$ $m\times n$-matriisi ja $B$ $n\times k$-matriisi. Osoita, että $(AB)^T=B^TA^T.$
\end{teht}

\begin{teht} Taso kulkee pisteiden $A=(1,2,1),\ B=(-1, 0,2)$ ja $C=(-3,1,3)$ kautta. Määrää tason normaalimuotoinen yhtälö käyttäen ristituloa.
\end{teht}
\begin{teht} Määrää ristitulon avulla yksikkövektori, joka on kohtisuorassa vektoreita $\bar{i}+\bar{j}$ ja $\bar{j}+3\bar{k}$ vastaan.
\end{teht}
\begin{teht} Määrää kolmion $ABC$ pinta-ala ristituloa käyttäen, kun $A=(1,2,0)$, $B=(1,0,2)$ ja $C=(0, 3, 1).$
\end{teht}
\begin{teht} Jos avaruuden kolme vektoria $\vec{u}, \vec{v}$ ja $\vec{w}$ on annettu. Millä ehdolla vektorit ovat samassa tasossa?
\end{teht}


\begin{teht} Avaruuden $\mathbf{R}^3$ vektorien $\vec{a},\ \vec{b}$ ja $\vec{c}$ skalaarikolmitulo on 
$$\vec{a}\bullet \vec{b}\times\vec{c}=\vec{a}\bullet (\vec{b}\times\vec{c}).$$
Osoita, että
$$
\vec{a}\bullet \vec{b}\times\vec{c}=\det \left(
\vec{a}, \vec{b}, \vec{c}
\right)
$$
\end{teht}

\begin{teht}
Osoita, että lineaarisella yhtälöryhmällä $A\vec{x}=\vec{b}$ ei voi olla täsmälleen kahta ratkaisua.
Vinkki: Tarkastele eri ratkaisujen välisen suoran pisteitä.
\end{teht}


\begin{teht} Fotosynteesissä kasvi muodostaa auringonvalosta saamallaan energialla hiilidioksista ja vedestä happea ja glukoosia. Etsi reaktion 
$$
x_1\chemfig{CO_2}+x_2\chemfig{H_2 O}\rightarrow x_3\chemfig{O_2}+x_4\chemfig{C_6 H_12 O_6}
$$
kertoimet $x_i,\ i=1, 2, 3, 4.$
\end{teht}

\begin{teht} Määrää homogeenisen yhtälöryhmän
$$\begin{cases}
-x+y-z-3w&=0\\
3x+y-z-w&=0\\
2x-y-2z-w&=0
\end{cases}
$$
ratkaisut.
\end{teht}

\begin{teht} Olkoon $L:\mathbf{R}^3\to \mathbf{R}^3$ lineaarikuvaus $L(\vec{x})=\vec{x}\times \vec{a},$ jossa $\vec{a}=(1, 2, -4).$ Määritä lineaarikuvauksen matriisi.

\end{teht}

\begin{teht} $m\times n$-matriisin $A$ nolla-avaruus eli ydin on
$$
\Null{A}=\{\vec{x}\in\mathbf{R}^n:\, A\vec{x}=0\}.
$$
Määritä matriisin
$$
A=\begin{pmatrix}
1 & 0 &0\\
0 & 1 & 0
\end{pmatrix}
$$
nolla-avaruus. Mikä on sen dimensio?
\end{teht}

\begin{teht} Olkoon
$$
A=\begin{pmatrix}
1 & 1 & 0 & 0 \\
0 & 1 & 1 &0\\
0 & 0 & 1 & 1\\
0 & 0 & 0 & 1
\end{pmatrix}.
$$
Laske $A^k$, kun $k=1, 2, 3, \ldots, 1992.$
\end{teht}

\begin{teht} Jos $A$ on neliömatriisi, niin yhtälöryhmä $A\vec{x}=\vec{b}$ on yksinkertaista ratkaista, jos
$A=LU$, missä $L$ on alakolmio- ja $U$ yläkolmiomatriisi. Ratkaise yhtälö
$$
A\vec{x}=\vec{b},
$$
jossa
$$
A=\begin{pmatrix}
3 & -7 & -2\\
-3 & 5 & 1\\
6 & -4 & 0
\end{pmatrix}
$$
ja $\vec{b}=(-7, 5, 2)^T$, käyttäen LU-hajotelmaa.
\end{teht}

\begin{teht}
Muodosta permutaatiomatriisi, joka siirtää $n\times n$-matriisissa kaikkia rivejä yhden ylöspäin, paitsi ylin rivi siirtyy alimmaiseksi. Tarkista saamasi matriisi tapauksessa, jossa 
$$
A=\begin{pmatrix}
2 & -1 & 0 & 12\\
1 & -1 & 0 & 1\\
0 & -11 & 1 & -2\\
0 & 1 & 1 & 1
\end{pmatrix}.
$$
\end{teht}

\begin{teht} Tarkestellaan tasossa $\mathbf{R}^2$ peilausta origon suhteen. Muodosta peilausmatriisi $Q$ ja laske sen ominaisarvot ja -vektorit.
\end{teht}

\begin{teht} Etsi matriisin
$$
A=\begin{pmatrix*}[r]
-3 & -2 & 4\\
-2 & 0 & 2\\
4 & 2 & -3
\end{pmatrix*}
$$
ortogonaalinen diagonalisointi.
\end{teht}

\begin{teht} Olkoon $A$ symmetrinen. Etsi muunnos, jonka avulla meliömuoto
$
q(\vec{x})=\vec{x}^TAx
$
saadaan pääakseliesitysmuotoon
$$
q(\vec{x})=\sum_{i=1}^n \lambda_i y_i^2.
$$
\end{teht}

\begin{teht} Määrää neliömuodon $q(\vec{x})=-3x_1^2-3x_3^2-4x_1x_2+8x_1x_3+4x_2x_3$ pääakseliesitys.
\end{teht}

\begin{teht} Luokittele toisen asteen käyrä $$
5x^2+5y^2-8xy-18x+18y+8=0
$$
ja etsi sen pääakseliesitys.

\end{teht}

\begin{teht} Määritä sen paraabelin yhtälö, joka parhaiten sopii (pienimmän neliösumman mielessä) pisteisiin $(-1, 1/2),\ (1, -1),\ (2, -1/2)$ ja $(3, 2).$
\end{teht}
\begin{teht} Määritä säämallin siirtymäkaavion siirtymämatriisi.

\begin{tikzpicture}[font=\sffamily]

    % Add the states
    \node[state,
          text=yellow,
          draw=none,
          fill=gray!50!black] (s) {Aurinkoista};
    \node[state,
          right=2cm of s,
          text=blue!30!white, 
          draw=none, 
          fill=gray!50!black] (r) {Sateista};

    % Connect the states with arrows
    \draw[every loop,
          auto=right,
          line width=1mm,
          >=latex,
          draw=orange,
          fill=orange]
        (s) edge[bend right, auto=left]  node {0{,}6} (r)
        (r) edge[bend right, auto=right] node {0{,}7} (s)
        (s) edge[loop above]             node {0{,}4} (s)
        (r) edge[loop above]             node {0{,}3} (r);
   \end{tikzpicture}

Määritä siirtymämatriisin tasapainojakauma.
\end{teht}
\end{document}
