\documentclass[12pt, a4paper, t]{beamer}
\usefonttheme[onlymath]{serif}
\setbeamertemplate{theorems}[ams style] 
\usepackage[utf8]{inputenc}
\usepackage{lmodern}
%\setbeamertemplate{frametitle}{\thesection.~\insertsection~\insertframetitle}

%\makeatletter
%\CheckCommand*\beamer@checkframetitle{\@ifnextchar\bgroup\beamer@inlineframetitle{}}
%\renewcommand*\beamer@checkframetitle{\global\let\beamer@frametitle\relax\@ifnextchar\bgroup\beamer@inlineframetitle{}}
%\makeatother

%\usepackage{paralist}
\usepackage[T1]{fontenc}
%\usepackage{charter}
\usepackage{mathdesign}

\usepackage[english, finnish]{babel}

%\usepackage[scaled]{helvet}
%\renewcommand*\familydefault{\sfdefault}
\usepackage{xcolor}


\usepackage{amsmath}
%\usepackage{amsfonts}    
\usepackage{amssymb}
\usepackage{mathtools}
\usepackage{amsthm}
\usepackage{xpatch}
\makeatletter
\xpatchcmd{\@thm}{\thm@headpunct{.}}{\thm@headpunct{}}{}{}
\makeatother
\theoremstyle{exercise}
\newtheorem{teht}{Tehtävä}
\theoremstyle{remark}
\newtheorem*{huom}{Huomautus}
\theoremstyle{definition}
%\newtheorem*{theorem}{Lause}
\newtheorem*{lause}{Lause}
\newtheorem*{maar}{Määritelmä}
\newtheorem{esim}{Esimerkki}
%\usepackage{aaltothesisoma}
%\usepackage{aaltologo}
%\usepackage{aaltothesis}
%\usepackage[agsm]{harvard}
%\renewcommand{\harvardand}{\&}
%\newcommand{\citee}[1]{(\citename*{#1} \citeyear*{#1}.)}
%\usepackage{ulem}  

\usepackage[natbib=true, bibstyle=authoryear, citestyle=authoryear]{biblatex}

%
%\addbibresource{/home/vsaariki/Latex/Bibliography/bibliography.bib}
%\usepackage{natbib}
%\bibliographystyle{/home/vsaariki/Latex/Bibliography/myabbrvnat}
%\bibliographystyle{agsm}
%\bibliographystyle{dcu}
%\bibliographystyle{abbrvnat}
%\bibliographystyle{plainnat}
%\bibliographystyle{unsrtnat}
%\renewcommand\harvardyearleft{\unskip\ }
%\renewcommand\harvardyearright[1]{.}
%\bibpunct{(}{)}{;}{a}{}{,~}

%\usepackage{wasysym}
%\usepackage{ccfonts,eulervm}
%Yksiköt
\usepackage[decimalsymbol=comma, load=prefixed]{siunitx}

\usepackage{datetime}
\renewcommand{\dateseparator}{.}
%\date{\dmyyyydate\today}
%\date{17.11.2014}

\usepackage{booktabs} % Tarjoaa \toprule-, \midrule- ja \bottomrule-komennot taulukkoja varten
	\renewcommand{\arraystretch}{1.3}

\usepackage{wasysym}
\usepackage{listing}
\usepackage{bbm}

\usepackage{tikz}
\usepackage{tikz-3dplot}
\usetikzlibrary{arrows}
\usepackage{lipsum}
\usepackage{tkz-euclide}
\usetikzlibrary{calc}
\usepackage[tikz]{bclogo}
\usepackage[framemethod=tikz]{mdframed}
\usepackage{sidecap}
\usetikzlibrary{calc,fadings,decorations.pathreplacing}
\usepackage{xcolor}
\usetikzlibrary{positioning}
\usetikzlibrary{through,backgrounds}
\usepackage{footmisc}

\usepackage{pdfpages}
\usepackage[bf,labelsep=period]{caption}


\usepackage{bm}
\usepackage{footmisc}

%\usepackage[bf,labelsep=period]{caption}


%\usepackage{caption}
\usepackage{subcaption}

\usepackage{tikz}
\usepackage{tikz-3dplot}

\usetikzlibrary{calc}
\usepackage{sidecap}
\usetikzlibrary{calc,fadings,decorations.pathreplacing}
\usepackage{xcolor}
\usetikzlibrary{positioning}
\usetikzlibrary{through,backgrounds}
\usepackage{footmisc}
\usetikzlibrary{decorations.pathmorphing,decorations.markings}


\usepackage{tcolorbox}% http://ctan.org/pkg/tcolorbox

\long\def\symbolfootnote[#1]#2{\begingroup%
\def\thefootnote{\fnsymbol{footnote}}\footnote[#1]{#2}\endgroup} 
% Use this if you run latex and use eps-format pictures
%\usepackage[dvips]{graphicx}

% Use this if you run pdflatex and use jpg/pdf-format pictures
% Note that margins are not correct with pdflatex output. 
%\usepackage[pdftex]{graphicx} 

%% Use this if you want to get links and nice output with pdflatex
%\usepackage[pdfpagemode=None,colorlinks=true,urlcolor=red, linkcolor=blue,citecolor=black,pdfstartview=FitH]{hyperref}

%% Use this if you do not like hyperref package - this
%% defines url environment and formats it correctly
\usepackage{url}

\usepackage{trace}
 




\usepackage{wrapfig}












% Mathematics
\newcommand{\vii}{\mathop{\Big/}}
\newcommand{\viiva}[2]{\vii\limits_{\!\!\!\!{#1}}^{\>\,{#2}}}
\newcommand{\relR}{\mathrel{\mathsf{R}}}
\newcommand{\cp}[1]{{#1}^\complement}
\newcommand{\raj}[2][f]{#1\!\!\mid_{#2}}
\renewcommand{\bar}[1]{\overline{#1}}
\newcommand{\vek}[1]{\mbox{\boldmath$#1$}}
\newcommand{\vekp}[1]{\hat{\vek{#1}}}
%\renewcommand{\vec}[1]{\vek{#1}}
%\usepackage{kpfonts}
\newcommand*{\vv}[1]{\overrightarrow{#1}}
\newcommand{\matr}[1]{\mathsf{#1}}
\newcommand{\E}{\mathrm{e}}
\newcommand{\I}{\mathrm{i}}
%\newcommand{\D}[1][x]{\,\text{d}#1}
%\newcommand{\km}{\,\mathrm{km}}
\newcommand{\abs}[1]{\lvert#1\rvert}
\newcommand{\norm}[1]{\lVert#1\rVert}
\newcommand{\mean}[1]{\langle#1\rangle}
\newcommand{\fneg}{f^{-1}}
\newcommand{\ftil}{\tilde{f}}
\newcommand{\ainaa}{{\alpha \in A}}
\newcommand{\INT}[1]{\overset{\mspace{6mu}\circ}{#1}}
\newcommand{\osit}[2]{\frac{\partial #1}{\partial #2}}
\newcommand{\diff}[2]{\frac{d #1}{d #2}}
\renewcommand{\angle}{\sphericalangle}
\newcommand{\epsi}{\varepsilon}
\newcommand{\epsii}{\epsi_\text{i}}
\newcommand{\epsie}{\epsi_\text{eff}}
\newcommand{\epsit}{\epsi_\text{t}}
\newcommand{\epsir}{\epsi_\text{r}}
\newcommand{\uz}{\vek{u}_z}
\newcommand{\roo}{\varrho}
\newcommand{\fii}{\phi_\text{i}}
\newcommand{\fit}{\phi_\text{t}}
\newcommand{\Ei}{\vek{E}_\text{i}}
\newcommand{\Et}{\vek{E}_\text{t}}
\newcommand{\pr}{\mathfrak{p}}
\renewcommand{\Re}[1]{\textrm{Re}\, #1}
\renewcommand{\Im}[1]{\textrm{Im}\, #1}
\DeclareMathOperator{\Arg}{Arg}
\DeclareMathOperator{\Ln}{Ln}
\newcommand{\phg}{p_\text{HG}}
\newcommand{\kB}{k_\textrm{B}}

\newcommand{\nub}{\bar{\nu}}

\newcommand{\bra}[1]{\ensuremath{\left\langle#1\right|}}
\newcommand{\ket}[1]{\ensuremath{\left|#1\right\rangle}}
\newcommand{\bracket}[2]{\ensuremath{\left\langle#1 \vphantom{#2}\right| \left. #2 \vphantom{#1}\right\rangle}}
\newcommand{\matrixel}[3]{\ensuremath{\left\langle #1 \vphantom{#2#3} \right| #2 \left| #3 \vphantom{#1#2} \right\rangle}}
\newcommand{\ipr}[1]{\ensuremath{\left\langle #1\right\rangle}}

\newcommand{\St}[1]{#1_\text{S}}
\newcommand{\aSt}[1]{#1_\text{a-S}}
\newcommand{\vib}[1]{#1_\text{vib}}

\DeclareMathOperator{\sinc}{sinc}
\DeclareMathOperator{\tr}{tr}
\DeclareMathOperator{\Exp}{Exp}
\DeclareMathOperator{\syt}{syt}
\DeclareMathOperator{\pyj}{pyj}
\DeclareMathOperator{\pym}{pym}

%\usepackage{enumitem}
%\usepackage[inline]{enumitem}
%\newenvironment{itemize*}%
%  {\begin{itemize}%
 %   \setlength{\itemsep}{0pt}%
 %   \setlength{\parskip}{0pt}}%
%  {\end{itemize}}
%\renewenvironment{itemize*}{\itemize}


%\newenvironment{enumerate*}%
 % {\begin{itemize}%
 %   \setlength{\itemsep}{0pt}%
 %   \setlength{\parskip}{0pt}}%
 % {\end{itemize}}
%\renewenvironment{enumerate*}{\enumerate}
\usepackage{enumitem}

\newenvironment{itemizea}%
  {\begin{itemize}%
    \setlength{\itemsep}{0pt}%
    \setlength{\parskip}{0pt}}%
  {\end{itemize}}
%\renewenvironment{itemize*}{\itemize}


\newenvironment{enumeratea}%
  {\begin{itemize}%
    \setlength{\itemsep}{0pt}%
    \setlength{\parskip}{0pt}}%
  {\end{itemize}}
%\renewenvironment{enumerate*}{\enumerate}

\newtheorem*{rat}{Ratkaisu}



\usepackage{pdfpages}
\addtobeamertemplate{block begin}{\vspace{-50pt}}{}

\usefonttheme{structurebold}
\usepackage{tasks}

    \setbeamerfont{title,frametitle}{series=\bfseries,parent=structure}
  \definecolor{oma}{rgb}{0.6, 0.25, 0.5}
  
\setbeamercolor{title}{fg=oma}
\definecolor{brightmaroon}{rgb}{0.76, 0.13, 0.28}
\setbeamercolor{frametitle}{fg=brightmaroon}

\global\mdfdefinestyle{exampledefault}{%
pstrickssetting={linestyle=dashed,},linecolor=red,middlelinewidth=1.5pt}

\definecolor{amaranth}{rgb}{0.9, 0.17, 0.31}

\definecolor{myframecolour}{HTML}{FF0013}

\addtobeamertemplate{block begin}{\vspace{-50pt}}{}

\usefonttheme{structurebold}
\usepackage{tcolorbox}% http://ctan.org/pkg/tcolorbox
    \tcbuselibrary{skins,breakable}
%\newtcolorbox{mybox}[1]{colback=red!5!white,colframe=red!75!black,fonttitle=\bfseries,title=#1}
\usepackage{pdfpages}

    \setbeamerfont{title,frametitle}{series=\bfseries,parent=structure}
  \definecolor{oma}{rgb}{0.6, 0.25, 0.5}
  
\setbeamercolor{title}{fg=oma}
\definecolor{brightmaroon}{rgb}{0.76, 0.13, 0.28}
\setbeamercolor{frametitle}{fg=brightmaroon}

\global\mdfdefinestyle{exampledefault}{%
pstrickssetting={linestyle=dashed,},linecolor=red,middlelinewidth=1.5pt}

\definecolor{amaranth}{rgb}{0.9, 0.17, 0.31}

\usepackage{tabularx}
\usepackage{booktabs}
\usepackage{colortbl}
%\usepackage{tikz}
\usetikzlibrary{calc}
\pgfdeclarelayer{background}
\pgfdeclarelayer{foreground}
\pgfsetlayers{background,main,foreground}
\setbeamertemplate{background canvas}[vertical shading]%
  [top=gray!1,bottom=gray!10]
\setbeamertemplate{navigation symbols}{}

\usepackage{tcolorbox}% http://ctan.org/pkg/tcolorbox
\newtcolorbox{mybox}[1]{colback=red!5!white,colframe=red!75!black,fonttitle=\bfseries,title=#1}
\definecolor{mybrown}{RGB}{128,64,0}
\newtcolorbox{myboxa}[1]{colback=mybrown!5!white,colframe=mybrown!100!black,fonttitle=\bfseries,title=#1}

\newenvironment{määr}[2]
{\begin{tcolorbox}[adjusted title=Määritelmä #1,colframe=gray!75!black]
%\begin{maar*}[''Jakaa täysin'']
#2
%\end{maar*}
\end{tcolorbox}
}

\tcbset{mytitle/.style={title={Määritelmä~\thetcbcounter\ifstrempty{#1}{}{ #1}}}}
\newtcolorbox[auto counter, number within=chapter, number freestyle={\noexpand\thechapter.\noexpand\arabic{\tcbcounter}}]{maari}[1][]{%
    enhanced,
    breakable,
   fonttitle=\bfseries,
    mytitle={},
    #1
}

\usetkzobj{all}
\date{}
\addtobeamertemplate{frametitle}{}{\vspace{-.6em}}
%\usepackage[inline]{enumitem}
  %\addtobeamertemplate{frametitle}{\vspace*{-.4cm}}{\vspace*{0.2cm}}
  \usetikzlibrary{calc,patterns,angles,quotes}
\begin{document}




%%%%%%%%%%%%%%%%%%%%%%%%%%%%%%%%%%%%%%%%%%%%%%%%%%%%%%%%%%%%%%%%%%%%%%%%%%%%%%%%%%%%%%%%%%%%%%%%%%%%%%%%%%%%%%%%%%%%%%%%%%%%%%%%%%%%%%%%%%%%%%%%%%%%%%%%%%%%%%%%%%%


%\input{vek_summa.tex}
%%%%%%%%%%%%%%%%%%%%%%%%%%%%%%%%%%%%%%%%%%%%%%%%%%%%%%%%%%%%%%%%%%%%%%%%%%%%%%%%%%%%%%%%%%%%%%%%%%%%%%%%%%%%%%%%%%%%%%%%%%%%%%%%%%%%%%%%%%%%%%%%%%%%%%%%%%%%%%%%%%%
\begin{frame}{Tehtävä 1}

\vspace{12pt}
\begin{tcolorbox}Määritä vektorien $\bar{a}=(1, 2, 3)$ ja $\bar{b}=(-1, 4, -2)$ pistetulo.\end{tcolorbox}
%\item Määrää kompleksiluvun $\displaystyle z=1-\sqrt{3}i$ eksponenttiesitys. e

\vspace{100pt}

\begin{rat}
%\begin{enumerate}[leftmargin=*, label=\alph*), topsep=0pt, itemsep=12pt, parsep=0pt, font=\bfseries\color{red}]
%\item 
Pistemerkintä ja sen paikkavektori samastetaan eli $\bar{a}=(1, 2, 3)=\bar{i}+2\bar{j}+3\bar{k}.$ Matriisilaskennassa vektori kirjoitetaan pystyvektorina eli
$
\bar{a}=\begin{pmatrix}
1\\
2\\
3
\end{pmatrix}=\begin{pmatrix}
1, 2, 3\end{pmatrix}^T.
$
Vektorien pistetulo on
$$
\bar{a}\cdot \bar{b}=\bar{a}^T\bar{b}=\begin{pmatrix} 1, 2, 3\end{pmatrix}\begin{pmatrix*}[r]
-1\\
4\\
-2
\end{pmatrix*}=1\cdot(-1)+2\cdot 4+3\cdot(-2)=1.
$$

%\pause
%\item Ratkaistaan
%\end{enumerate}
\end{rat}
\end{frame}


\begin{frame}{Tehtävä 2}

\vspace{12pt}
\begin{tcolorbox} Laske matriisin $A$ ja vektorin $\bar{x}$ tulo, kun
$$
A=\begin{pmatrix*}[r]
1 & 2 & 3\\
-1 & 4 & -2\\
1 & 0&-1\\
\end{pmatrix*}\quad\text{ja}\quad \bar{x}=\begin{pmatrix*}[r]
1\\
-2\\
-1
\end{pmatrix*}.
$$.\end{tcolorbox}
%\item Määrää kompleksiluvun $\displaystyle z=1-\sqrt{3}i$ eksponenttiesitys. e

\vspace{100pt}

\begin{rat}
%\begin{enumerate}[leftmargin=*, label=\alph*), topsep=0pt, itemsep=12pt, parsep=0pt, font=\bfseries\color{red}]
%\item 
Matriisi $A$ on $3\times 3$-matriisi, jossa ensimmäinen kolmonen viittaa rivien (vaakarivien) lukumäärään ja jälkimmäinen kolmonen sarakkeiden (pystyrivien) lukumäärään. Vektori $\bar{x}$ on $3\times 1$-matriisi eli lyhyesti vain $3$-komponenttinen vektori. Tulo $A\bar{x}$ on määritelty ja tulo on $3$-komponenttinen vektori.
\end{rat}

\end{frame}
\begin{frame}
\vspace{100pt}
\begin{rat}[jatkuu]
Tulo $A\bar{x}$ lasketaan niin, että lasketaan $A$:n sarakevektorien ja vektorin $\bar{x}$ pistetulot ja ne tulevat tulovektorin komponenteiksi.
\begin{align*}
A\bar{x}&=\begin{pmatrix*}[r]
1 & 2 & 3\\
-1 & 4 & -2\\
1 & 0&-1\\
\end{pmatrix*}\begin{pmatrix*}[r]
1\\
-2\\
-1
\end{pmatrix*}\\
&=\begin{pmatrix*}[r]
1\cdot 1+2\cdot(-2)+3\cdot(-1)\\
-1 \cdot 1+	4\cdot (-2)-2\cdot(-1)\\
1 \cdot 1+0\cdot(-2)-1\cdot(-1)\\
\end{pmatrix*}\\
&=\begin{pmatrix*}[r]
-6\\
-7\\
2
\end{pmatrix*}.\\
\end{align*}

\end{rat}

\end{frame}



\begin{frame}{Tehtävä 3}

\vspace{12pt}
\begin{tcolorbox} Laske matriisitulo $AB$, kun
$$
A=\begin{pmatrix*}[r]
1 & 2 & 3\\
-1 & 4 & -2\\
1 & 0&-1\\
\end{pmatrix*}\quad\text{ja}\quad B=\begin{pmatrix*}[r]
1 & 2 & 3 &4\\
-2 & -1 & 0 & 2\\
-1 & 1 & 2 &-2
\end{pmatrix*}.
$$.\end{tcolorbox}
%\item Määrää kompleksiluvun $\displaystyle z=1-\sqrt{3}i$ eksponenttiesitys. e

\vspace{100pt}

\begin{rat}
%\begin{enumerate}[leftmargin=*, label=\alph*), topsep=0pt, itemsep=12pt, parsep=0pt, font=\bfseries\color{red}]
%\item 
Varmistetaan aluksi, että tulo voidaan laskea. Matriisi $A$ on $3\times 3$-matriisi ja $B$ on $3\times 4$-matriisi, joten $AB$ on määritelty, koska matriisissa $A$ on yhtä monta saraketta kuin matriisissa $B$ on rivejä. Tulo $AB$ on $3\times 4$-matriisi. Tulomatriisin $AB$ alkio, joka on $i$:nnellä rivillä ja $j$:nnellä sarakkeella, lasketaan $A$:n $i$:nnen rivivektorin ja $B$:n $j$:nnen sarakevektorin pistetulona.
\end{rat}

\end{frame}


\begin{frame}
\vspace{100pt}
\begin{rat}[jatkuu]
Lasketaan pistetulot
\begin{align*}
\begin{pmatrix*}[r]
1 & 2 &3
\end{pmatrix*}
\begin{pmatrix*}[r]
1\\
-2\\
-1
\end{pmatrix*}&=-6,
\begin{pmatrix*}[r]
1 & 2 &3
\end{pmatrix*}
\begin{pmatrix*}[r]
2\\
-1\\
1
\end{pmatrix*}=3,\\
\begin{pmatrix*}[r]
1 & 2 &3
\end{pmatrix*}
\begin{pmatrix*}[r]
3\\
0\\
2
\end{pmatrix*}&=9,
\begin{pmatrix*}[r]
1 & 2 &3
\end{pmatrix*}
\begin{pmatrix*}[r]
4\\
2\\
-2
\end{pmatrix*}= 2, \\
\begin{pmatrix*}[r]
-1 & 4 &-2
\end{pmatrix*}
\begin{pmatrix*}[r]
1\\
-2\\
-1
\end{pmatrix*}&=-7 ,
\begin{pmatrix*}[r]
-1 & 4 &-2
\end{pmatrix*}
\begin{pmatrix*}[r]
2\\
-1\\
1
\end{pmatrix*}=-8,\\
\end{align*}

\end{rat}

\end{frame}
\begin{frame}
\vspace{90pt}
\begin{rat}[jatkuu]
\begin{align*}
\begin{pmatrix*}[r]
-1 & 4 &-2
\end{pmatrix*}
\begin{pmatrix*}[r]
3\\
0\\
2
\end{pmatrix*}&=-7, 
\begin{pmatrix*}[r]
-1 & 4 &-2
\end{pmatrix*}
\begin{pmatrix*}[r]
4\\
2\\
-2
\end{pmatrix*}=-8,\\
\begin{pmatrix*}[r]
-1 & 4 &-2
\end{pmatrix*}
\begin{pmatrix*}[r]
3\\
0\\
2
\end{pmatrix*}&=-7,
\begin{pmatrix*}[r]
-1 & 4 &-2
\end{pmatrix*}
\begin{pmatrix*}[r]
4\\
2\\
-2
\end{pmatrix*}=8,\\
\begin{pmatrix*}[r]
1 & 0 &-1
\end{pmatrix*}
\begin{pmatrix*}[r]
1\\
-2\\
-1
\end{pmatrix*}&=2, 
\begin{pmatrix*}[r]
1 & 0 &-1
\end{pmatrix*}
\begin{pmatrix*}[r]
2\\
-1\\
1
\end{pmatrix*}=1,\\
\end{align*}

\end{rat}

\end{frame}

\begin{frame}
\vspace{90pt}
\begin{rat}[jatkuu aina vaan...]
\begin{align*}
\begin{pmatrix*}[r]
1 & 0 &-1\\
\end{pmatrix*}
\begin{pmatrix*}[r]
3\\
0\\
2
\end{pmatrix*}&=1, \begin{pmatrix*}[r]
1 & 0 &-1
\end{pmatrix*}
\begin{pmatrix*}[r]
4\\
2\\
-2
\end{pmatrix*}=6.
\end{align*}
Ikävän puurtamisen jälkeen saadaan 
$$
AB=\begin{pmatrix*}[r]
-6 & 3 & 9 & 2\\
-7 & -8 & -7 & 8\\
2 & 1 & 1& 6
\end{pmatrix*}.
$$
\end{rat}
\end{frame}

\end{document}